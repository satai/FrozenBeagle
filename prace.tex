%%% Hlavní soubor. Zde se definují základní parametry a odkazuje se na ostatní části. %%%

%% Verze pro jednostranný tisk:
% Okraje: levý 40mm, pravý 25mm, horní a dolní 25mm
% (ale pozor, LaTeX si sám přidává 1in)
\documentclass[12pt,a4paper]{report}
\setlength\textwidth{145mm}
\setlength\textheight{247mm}
\setlength\oddsidemargin{15mm}
\setlength\evensidemargin{15mm}
\setlength\topmargin{0mm}
\setlength\headsep{0mm}
\setlength\headheight{0mm}
% \openright zařídí, aby následující text začínal na pravé straně knihy
\let\openright=\clearpage

%% Pokud tiskneme oboustranně:
% \documentclass[12pt,a4paper,twoside,openright]{report}
% \setlength\textwidth{145mm}
% \setlength\textheight{247mm}
% \setlength\oddsidemargin{15mm}
% \setlength\evensidemargin{0mm}
% \setlength\topmargin{0mm}
% \setlength\headsep{0mm}
% \setlength\headheight{0mm}
% \let\openright=\cleardoublepage

%% Pokud používáte csLaTeX (doporučeno):
\usepackage{czech}
%% Pokud nikoliv:
%\usepackage[czech]{babel}
%\usepackage[T1]{fontenc}

%% Použité kódování znaků: obvykle latin2, cp1250 nebo utf8:
\usepackage[utf8]{inputenc}

%% Ostatní balíčky
\usepackage{graphicx}
\usepackage{amsthm}

%% Balíček hyperref, kterým jdou vyrábět klikací odkazy v PDF,
%% ale hlavně ho používáme k uložení metadat do PDF (včetně obsahu).
%% POZOR, nezapomeňte vyplnit jméno práce a autora.
\usepackage[ps2pdf,unicode]{hyperref}   % Musí být za všemi ostatními balíčky
\hypersetup{pdftitle=Název práce}
\hypersetup{pdfauthor=Jméno Příjmení}

%%% Drobné úpravy stylu

% Tato makra přesvědčují mírně ošklivým trikem LaTeX, aby hlavičky kapitol
% sázel příčetněji a nevynechával nad nimi spoustu místa. Směle ignorujte.
\makeatletter
\def\@makechapterhead#1{
  {\parindent \z@ \raggedright \normalfont
   \Huge\bfseries \thechapter. #1
   \par\nobreak
   \vskip 20\p@
}}
\def\@makeschapterhead#1{
  {\parindent \z@ \raggedright \normalfont
   \Huge\bfseries #1
   \par\nobreak
   \vskip 20\p@
}}
\makeatother

% Toto makro definuje kapitolu, která není očíslovaná, ale je uvedena v obsahu.
\def\chapwithtoc#1{
\chapter*{#1}
\addcontentsline{toc}{chapter}{#1}
}

\begin{document}

% Trochu volnější nastavení dělení slov, než je default.
\lefthyphenmin=2
\righthyphenmin=2

%%% Titulní strana práce

\pagestyle{empty}
\begin{center}

\large

Univerzita Karlova v Praze

\medskip

Přírodovědná fakulta

\vfill

{\bf\Large DIPLOMOVÁ PRÁCE}


%nedame tam logo
%\vfill

%\centerline{\mbox{\includegraphics[width=60mm]{logo.eps}}}

\vfill
\vspace{5mm}

{\LARGE Ondřej Nekola}

\vspace{15mm}

% Název práce přesně podle zadání
{\LARGE\bfseries Simulace dějů predikovaných teorií zamrzlé plasticity}

\vfill

% Název katedry nebo ústavu, kde byla práce oficiálně zadána
% (dle Organizační struktury MFF UK)
Katedra filosofie a dějin přírodních věd

\vfill

\begin{tabular}{rl}

Vedoucí diplomové práce: & Prof. RNDr. Jaroslav Flegr, PhD.\\
\noalign{\vspace{2mm}}
Studijní program: & Teoretická a evoluční biologie \\
%\noalign{\vspace{2mm}}
%Studijní obor: & obor \\
\end{tabular}

\vfill

% Zde doplňte rok
Praha \the\year

\end{center}

\newpage

%%% Následuje vevázaný list -- kopie podepsaného "Zadání diplomové práce".
%%% Toto zadání NENÍ součástí elektronické verze práce, nescanovat.

%%% Na tomto místě mohou být napsána případná poděkování (vedoucímu práce,
%%% konzultantovi, tomu, kdo zapůjčil software, literaturu apod.)

\openright

\noindent
Děkuji rodičům a Julii za trpělivost a profesoru Flegerovi za způsob, jakým se mne ujal. Také bych rád vyjádřil své díky pražírnám Doubleshot a Bonanza, bez jejich laskavého pražení by tato práce nevznikla.

\newpage

%%% Strana s čestným prohlášením k diplomové práci

\vglue 0pt plus 1fill

\noindent
Tímto čestně prohlašuji, že jsem tuto diplomovou práci zpracoval samostatně, pod vedením prof. RNDr. Jaroslava Flegra, PhD., a že jsem uvedl a citoval veškeré použité informační zdroje. Tato práce ani žádná její část nebyla použita k získání žádného jiného akademického titulu.

\vspace{10mm}

\hbox{\hbox to 0.5\hsize{%
V Praze dne \today
\hss}\hbox to 0.5\hsize{%
    ..........
\hss}}

\vspace{20mm}
\newpage

%%% Povinná informační strana diplomové práce

\vbox to 0.5\vsize{ \setlength\parindent{0mm}
\setlength\parskip{5mm}

Název práce:
Simulace dějů predikovaných teorií zamrzlé plasticity
% přesně dle zadání

Autor:
Ondřej Nekola

Katedra:  % Případně Ústav:
Katedra filosofie a dějin přírodních věd
% dle Organizační struktury MFF UK

Vedoucí diplomové práce:
Prof. RNDr. Jaroslav Flegr, PhD.
% dle Organizační struktury MFF UK, případně plný název pracoviště mimo MFF UK

Abstrakt:
% abstrakt v rozsahu 80-200 slov; nejedná se však o opis zadání diplomové práce

Klíčová slova:
% 3 až 5 klíčových slov

\vss}\nobreak\vbox to 0.49\vsize{
\setlength\parindent{0mm}
\setlength\parskip{5mm}

Title:
Simulation of processes predicted by the frozen plasticity theory

Author:
Ondřej Nekola

Department:
Department of Philosophy and History of Science

Supervisor:
Prof. RNDr. Jaroslav Flegr, PhD.
% dle Organizační struktury MFF UK, případně plný název pracoviště
% mimo MFF UK v angličtině

Abstract:
% abstrakt v rozsahu 80-200 slov v angličtině; nejedná se však o překlad
% zadání diplomové práce

Keywords:
% 3 až 5 klíčových slov v angličtině

\vss}

\newpage

%%% Strana s automaticky generovaným obsahem diplomové práce. U matematických
%%% prací je přípustné, aby seznam tabulek a zkratek, existují-li, byl umístěn
%%% na začátku práce, místo na jejím konci.

\openright
\pagestyle{plain}
\setcounter{page}{1}
\tableofcontents

%%% Jednotlivé kapitoly práce jsou pro přehlednost uloženy v samostatných souborech
\include{uvod}

\include{0_Literarni_prehled}

\chapter{Metodika}

\section{Simulace}

V této části buďte velmi precizní. Vše musíte popsat tak důkladně, aby kdokoliv mohl vaše pokusy či pozorování zopakovat. Nezapomeňte uvést velikost vzorků, věk a pohlaví zvířat, denní a roční dobu, podmínky chovu nebo popis lokalit (třeba včetně mapky), specifické přístrojové vybavení (pokud je to důležité, např. pro srovnání výsledků s publikovanými údaji, uveďte i přesný typ přístroje) a jiné podrobnosti. Také může být dobré zmínit, jak jste zabránili vlivu pakovaného testování stejných
individuí a proč se domníváte, že je počet jedinců dostatečný k zodpovězení vašich otázek. Více než vhodné je zmínit a zdůvodnit použité statistické metody a počítačové programy. Používáte-li zkratky, uveďte jejich seznam.
Materiál a metodika bývají pro větší srozumitelnost často členěny na menší podkapitoly: materiál, experimentální design, analýza dat atd. Systém více podkapitol třeba jen o třech řádcích bývá přehlednější než souvislý odstavec na dvě strany. (A to samozřejmě neplatí jen pro metodiku.) Čtení DP usnadní, pokud je členění na podkapitoly obdobné v kapitolách Materiál a metodika, Výsledky a případně i Diskuse.
Při vší pečlivosti se však snažte být poměrně struční. Materiál a metodika by neměly tvořit většinu DP.

\section{Frozen Beagle}
Název první kapitoly
\section{Statistická analýza}
\section{Reprodukovatelnost}


\include{2_Vysledky}

\include{3_Diskuse}

\include{zaver}

%%% Seznam použité literatury
%%% Seznam použité literatury je zpracován podle platných standardů. Povinnou citační
%%% normou pro diplomovou práci je ISO 690. Jména časopisů lze uvádět zkráceně, ale jen
%%% v kodifikované podobě. Všechny použité zdroje a prameny musí být řádně citovány.

\def\bibname{Seznam použité literatury}
\begin{thebibliography}{xxxxx--99}
\addcontentsline{toc}{chapter}{\bibname}

\bibitem[Lamport--94]{lamport94}
  {\sc Lamport,} Leslie.
  \emph{\LaTeX: A Document Preparation System}.
  2. vydání.
  Massachusetts: Addison Wesley, 1994.
  ISBN 0-201-52983-1.

\bibitem[Haskell]{haskell}

\end{thebibliography}


%%% Tabulky v diplomové práci, existují-li.
\chapwithtoc{Seznam tabulek}

%%% Použité zkratky v diplomové práci, existují-li, včetně jejich vysvětlení.
\chapwithtoc{Seznam použitých zkratek}

%%% Přílohy k diplomové práci, existují-li (různé dodatky jako výpisy programů,
%%% diagramy apod.). Každá příloha musí být alespoň jednou odkazována z vlastního
%%% textu práce. Přílohy se číslují.
\chapwithtoc{Přílohy}

\openright
\end{document}
