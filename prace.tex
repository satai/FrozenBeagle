%%% Hlavní soubor. Zde se definují základní parametry a odkazuje se na ostatní části. %%%

%% Verze pro jednostranný tisk:
% Okraje: levý 40mm, pravý 25mm, horní a dolní 25mm
% (ale pozor, LaTeX si sám přidává 1in)
\documentclass[12pt,a4paper]{report}
\setlength\textwidth{145mm}
\setlength\textheight{247mm}
\setlength\oddsidemargin{15mm}
\setlength\evensidemargin{15mm}
\setlength\topmargin{0mm}
\setlength\headsep{0mm}
\setlength\headheight{0mm}
% \openright zařídí, aby následující text začínal na pravé straně knihy
\let\openright=\clearpage

%% Pokud tiskneme oboustranně:
% \documentclass[12pt,a4paper,twoside,openright]{report}
% \setlength\textwidth{145mm}
% \setlength\textheight{247mm}
% \setlength\oddsidemargin{15mm}
% \setlength\evensidemargin{0mm}
% \setlength\topmargin{0mm}
% \setlength\headsep{0mm}
% \setlength\headheight{0mm}
% \let\openright=\cleardoublepage

%% Pokud používáte csLaTeX (doporučeno):
\usepackage{czech}
%% Pokud nikoliv:
%\usepackage[czech]{babel}
%\usepackage[T1]{fontenc}

%% Použité kódování znaků: obvykle latin2, cp1250 nebo utf8:
\usepackage[utf8]{inputenc}

%% Ostatní balíčky
\usepackage{graphicx}
\usepackage{amsthm}

%% Balíček hyperref, kterým jdou vyrábět klikací odkazy v PDF,
%% ale hlavně ho používáme k uložení metadat do PDF (včetně obsahu).
%% POZOR, nezapomeňte vyplnit jméno práce a autora.
\usepackage[ps2pdf,unicode]{hyperref}   % Musí být za všemi ostatními balíčky
\hypersetup{pdftitle=Název práce}
\hypersetup{pdfauthor=Jméno Příjmení}

%%% Drobné úpravy stylu

% Tato makra přesvědčují mírně ošklivým trikem LaTeX, aby hlavičky kapitol
% sázel příčetněji a nevynechával nad nimi spoustu místa. Směle ignorujte.
\makeatletter
\def\@makechapterhead#1{
  {\parindent \z@ \raggedright \normalfont
   \Huge\bfseries \thechapter. #1
   \par\nobreak
   \vskip 20\p@
}}
\def\@makeschapterhead#1{
  {\parindent \z@ \raggedright \normalfont
   \Huge\bfseries #1
   \par\nobreak
   \vskip 20\p@
}}
\makeatother

% Toto makro definuje kapitolu, která není očíslovaná, ale je uvedena v obsahu.
\def\chapwithtoc#1{
\chapter*{#1}
\addcontentsline{toc}{chapter}{#1}
}

\begin{document}

% Trochu volnější nastavení dělení slov, než je default.
\lefthyphenmin=2
\righthyphenmin=2

%%% Titulní strana práce

\pagestyle{empty}
\begin{center}

\large

Univerzita Karlova v Praze

\medskip

Přírodovědná fakulta

\vfill

{\bf\Large DIPLOMOVÁ PRÁCE}


%nedame tam logo
%\vfill

%\centerline{\mbox{\includegraphics[width=60mm]{logo.eps}}}

\vfill
\vspace{5mm}

{\LARGE Ondřej Nekola}

\vspace{15mm}

% Název práce přesně podle zadání
{\LARGE\bfseries Simulace dějů predikovaných teorií zamrzlé plasticity}

\vfill

% Název katedry nebo ústavu, kde byla práce oficiálně zadána
% (dle Organizační struktury MFF UK)
Katedra filosofie a dějin přírodních věd

\vfill

\begin{tabular}{rl}

Vedoucí diplomové práce: & Prof. RNDr. Jaroslav Flegr, PhD.\\
\noalign{\vspace{2mm}}
Studijní program: & Teoretická a evoluční biologie \\
%\noalign{\vspace{2mm}}
%Studijní obor: & obor \\
\end{tabular}

\vfill

% Zde doplňte rok
Praha \the\year

\end{center}

\newpage

%%% Následuje vevázaný list -- kopie podepsaného "Zadání diplomové práce".
%%% Toto zadání NENÍ součástí elektronické verze práce, nescanovat.

%%% Na tomto místě mohou být napsána případná poděkování (vedoucímu práce,
%%% konzultantovi, tomu, kdo zapůjčil software, literaturu apod.)

\openright

\noindent
Děkuji rodičům a Julii za trpělivost a profesoru Flegerovi za způsob, jakým se mne ujal. Také bych rád vyjádřil své díky pražírnám Doubleshot a Bonanza, bez jejich laskavého pražení by tato práce nevznikla.

\newpage

%%% Strana s čestným prohlášením k diplomové práci

\vglue 0pt plus 1fill

\noindent
Tímto čestně prohlašuji, že jsem tuto diplomovou práci zpracoval samostatně, pod vedením prof. RNDr. Jaroslava Flegra, PhD., a že jsem uvedl a citoval veškeré použité informační zdroje. Tato práce ani žádná její část nebyla použita k získání žádného jiného akademického titulu.

\vspace{10mm}

\hbox{\hbox to 0.5\hsize{%
V Praze dne \today
\hss}\hbox to 0.5\hsize{%
    ..........
\hss}}

\vspace{20mm}
\newpage

%%% Povinná informační strana diplomové práce

\vbox to 0.5\vsize{ \setlength\parindent{0mm}
\setlength\parskip{5mm}

Název práce:
Simulace dějů predikovaných teorií zamrzlé plasticity
% přesně dle zadání

Autor:
Ondřej Nekola

Katedra:  % Případně Ústav:
Katedra filosofie a dějin přírodních věd
% dle Organizační struktury MFF UK

Vedoucí diplomové práce:
Prof. RNDr. Jaroslav Flegr, PhD.
% dle Organizační struktury MFF UK, případně plný název pracoviště mimo MFF UK

Abstrakt:
% abstrakt v rozsahu 80-200 slov; nejedná se však o opis zadání diplomové práce

Klíčová slova:
% 3 až 5 klíčových slov

\vss}\nobreak\vbox to 0.49\vsize{
\setlength\parindent{0mm}
\setlength\parskip{5mm}

Title:
Simulation of processes predicted by the frozen plasticity theory

Author:
Ondřej Nekola

Department:
Department of Philosophy and History of Science

Supervisor:
Prof. RNDr. Jaroslav Flegr, PhD.
% dle Organizační struktury MFF UK, případně plný název pracoviště
% mimo MFF UK v angličtině

Abstract:
% abstrakt v rozsahu 80-200 slov v angličtině; nejedná se však o překlad
% zadání diplomové práce

Keywords:
% 3 až 5 klíčových slov v angličtině

\vss}

\newpage

%%% Strana s automaticky generovaným obsahem diplomové práce. U matematických
%%% prací je přípustné, aby seznam tabulek a zkratek, existují-li, byl umístěn
%%% na začátku práce, místo na jejím konci.

\openright
\pagestyle{plain}
\setcounter{page}{1}
\tableofcontents

%%% Jednotlivé kapitoly práce jsou pro přehlednost uloženy v samostatných souborech
\chapter*{Úvod}
\addcontentsline{toc}{chapter}{Úvod}

Následuje několik ukázkových kapitol, které doporučují, jak by se
měla diplomová práce sázet. Primárně popisují použití \TeX{}ové
šablony, ale obecné rady poslouží dobře i~uživatelům jiných
systémů.


\chapter{Literární přehled}

\section{Název první podkapitoly v druhé kapitole}

\section{Název druhé podkapitoly v druhé kapitole}



\chapter{Metodika}

\section{Simulace}

V této části buďte velmi precizní. Vše musíte popsat tak důkladně, aby kdokoliv mohl vaše pokusy či pozorování zopakovat. Nezapomeňte uvést velikost vzorků, věk a pohlaví zvířat, denní a roční dobu, podmínky chovu nebo popis lokalit (třeba včetně mapky), specifické přístrojové vybavení (pokud je to důležité, např. pro srovnání výsledků s publikovanými údaji, uveďte i přesný typ přístroje) a jiné podrobnosti. Také může být dobré zmínit, jak jste zabránili vlivu pakovaného testování stejných
individuí a proč se domníváte, že je počet jedinců dostatečný k zodpovězení vašich otázek. Více než vhodné je zmínit a zdůvodnit použité statistické metody a počítačové programy. Používáte-li zkratky, uveďte jejich seznam.
Materiál a metodika bývají pro větší srozumitelnost často členěny na menší podkapitoly: materiál, experimentální design, analýza dat atd. Systém více podkapitol třeba jen o třech řádcích bývá přehlednější než souvislý odstavec na dvě strany. (A to samozřejmě neplatí jen pro metodiku.) Čtení DP usnadní, pokud je členění na podkapitoly obdobné v kapitolách Materiál a metodika, Výsledky a případně i Diskuse.
Při vší pečlivosti se však snažte být poměrně struční. Materiál a metodika by neměly tvořit většinu DP.

\section{Frozen Beagle}
Název první kapitoly
\section{Statistická analýza}
\section{Reprodukovatelnost}


\chapter{Výsledky}

V DP uveďte všechny zjištěné výsledky včetně výsledků pilotních experimentů. Důkladně však zvažte, jaká část vašich výsledků má být prezentována v kapitole Výsledky. Měly by tu být především výsledky, které čímsi přispívají k zodpovězení vašich otázek. Základní data (např. tabulky naměřených rozměrů nebo genotypy jedinců)  patří spíše do Příloh nebo na přiložené CD. 

Mějte na paměti, že obrázky spíše vyjadřují myšlenky, zatímco tabulky zobrazují data. Údaje z tabulek neuvádějte znovu v textu. Pozor také na duplikaci údajů obsažených v tabulkách a grafech. Každá tabulka a graf však musí být v textu zmíněny (zjednodušeně řečeno: tabulka ukáže data, která jsou dále okomentována v textu).

Příklad: V tabulce číslo 1 jsou naměřené délky žížal v cm (10  10,1  10,7  10,9  11).  V textu se pak například objeví: ...z Tabulky 1 vyplývá, že žížaly měřily od 10 do 11cm...

Tabulky i obrázky by měly být očíslovány a podle čísel taky seřazeny (tedy Tabulka 1 se v textu objeví dříve než Tabulka 2).

Věnujte také zvýšenou pozornost popiskům obrázků a tabulek, které musí být \uv{self-explanatoryi}  a zkontrolujte popis os v grafech. Tabulky by měly být co nejjednodušší. Vertikální čáry vnich nejlépe nepoužívejte vůbec a počet čar horizontálních omezte na minimum.

Musí být zřejmé, nejen které statistické testy byly použity, ale také zda jsou pro ně splněny předpoklady (např. normální rozložení, pokud to test vyžaduje). 

Samozřejmě můžete použít barevné grafy či tabulky. Je ale dobré si uvědomit, že se hodnotí obsah, nikoliv barevnost. Klidně tedy vystačíte i jen s černou a bílou barvou (to samozřejmě nemusí platit pro fotografie nebo obrázky). Pokud použijete barevné grafy, měly by být (pokud možno) rozlišitelné i v černobílém provedení.

\section{Název první podkapitoly v druhé kapitole}

\section{Název druhé podkapitoly v druhé kapitole}



\chapter{Diskuse}

Diskusi nepodceňujte. Je to nejdůležitější část a podle toho taky nejsložitější na napsání. Může se stát, že po napsání Diskuse budete muset přepsat třeba celý Úvod. Napsání kvalitní Diskuse vyžaduje spoustu času. Počítejte s tím. Za jeden večer to nenapíšete.

Citát: \uv{Z krátké diskuse čiší stupidita autora, který vlastně neví co diskutovat.} Dr. B. Mandák, botanický ústav AVČR

Bývá dobré začít shrnutím a interpretací vašich výsledků. Diskuse ale musí také ukázat jak výsledky zapadají do toho, co je o dané problematice známo. Musíte oddiskutovat jak soulad získaných výsledků s publikovanými tak i jejich nesoulad. Domníváte-li se, že jsou vaše výsledky zcela nové, pak vysvětlete v čem je jejich originalita. Pokud je nesoulad mezi výsledky vašimi a jiných badatelů, pak je nutné vysvětlit, čím k tomu mohlo dojít.

Příklad: Sonntag (2001) sice uvádí, že by žížala mohla být i celkem krátká, neměl však k dispozici přesné měřítko, kterým žížalu měřím já.

Nezapomeňte, že jste si v Úvodu vytyčili otázky a cíle. Na ně je třeba reagovat. Můžete i naznačit, jakým směrem by se teď po vašich zásadních objevech měl ubírat další výzkum. Můžete i formulovat nové hypotézy, které by měly být v budoucnu testovány. Pořadí diskutovaných okruhů by mělo být stejné jako bylo v Úvodu. Čtenář DP by měl být na Diskusi připraven Úvodem. Nemělo by se tedy stát, že se v Diskusi zjeví překvapivá fakta z prací jiných autorů, o nichž není v Úvodu ani zmínky. 

\section{Název první podkapitoly v druhé kapitole}

\section{Název druhé podkapitoly v druhé kapitole}



\chapter*{Závěr}

Stručně uveďte, co nového vaše DP přináší. Zmiňte jen ty nejdůležitější výsledky. Jedná se o souhrn výsledků. Neopakujte tedy fakta z Materiálu a metodiky.  Po přečtení souhrnu by mělo být každému zcela jasné, o čem vaše DP pojednává, aniž by četl jakýkoliv jiný text. Je slušné se vejít na jednu stránku. 

\addcontentsline{toc}{chapter}{Závěr}


%%% Seznam použité literatury
%%% Seznam použité literatury je zpracován podle platných standardů. Povinnou citační
%%% normou pro diplomovou práci je ISO 690. Jména časopisů lze uvádět zkráceně, ale jen
%%% v kodifikované podobě. Všechny použité zdroje a prameny musí být řádně citovány.

\def\bibname{Seznam použité literatury}
\begin{thebibliography}{xxxxx--99}
\addcontentsline{toc}{chapter}{\bibname}

\bibitem[Lamport--94]{lamport94}
  {\sc Lamport,} Leslie.
  \emph{\LaTeX: A Document Preparation System}.
  2. vydání.
  Massachusetts: Addison Wesley, 1994.
  ISBN 0-201-52983-1.

\end{thebibliography}


%%% Tabulky v diplomové práci, existují-li.
\chapwithtoc{Seznam tabulek}

%%% Použité zkratky v diplomové práci, existují-li, včetně jejich vysvětlení.
\chapwithtoc{Seznam použitých zkratek}

%%% Přílohy k diplomové práci, existují-li (různé dodatky jako výpisy programů,
%%% diagramy apod.). Každá příloha musí být alespoň jednou odkazována z vlastního
%%% textu práce. Přílohy se číslují.
\chapwithtoc{Přílohy}

\openright
\end{document}
