%%% Hlavní soubor. Zde se definují základní parametry a odkazuje se na ostatní části. %%%

%% Verze pro jednostranný tisk:
% Okraje: levý 40mm, pravý 25mm, horní a dolní 25mm
% (ale pozor, LaTeX si sám přidává 1in)
\documentclass[12pt,a4paper]{report}
\setlength\textwidth{145mm}
\setlength\textheight{247mm}
\setlength\oddsidemargin{15mm}
\setlength\evensidemargin{15mm}
\setlength\topmargin{0mm}
\setlength\headsep{0mm}
\setlength\headheight{0mm}
% \openright zařídí, aby následující text začínal na pravé straně knihy
\let\openright=\clearpage

%% Pokud tiskneme oboustranně:
% \documentclass[12pt,a4paper,twoside,openright]{report}
% \setlength\textwidth{145mm}
% \setlength\textheight{247mm}
% \setlength\oddsidemargin{15mm}
% \setlength\evensidemargin{0mm}
% \setlength\topmargin{0mm}
% \setlength\headsep{0mm}
% \setlength\headheight{0mm}
% \let\openright=\cleardoublepage

%% Pokud používáte csLaTeX (doporučeno):
\usepackage{czech}
%% Pokud nikoliv:
%\usepackage[czech]{babel}
%\usepackage[T1]{fontenc}

%% Použité kódování znaků: obvykle latin2, cp1250 nebo utf8:
\usepackage[utf8]{inputenc}

%% Ostatní balíčky
\usepackage{graphicx}
\usepackage{amsthm}

%% Balíček hyperref, kterým jdou vyrábět klikací odkazy v PDF,
%% ale hlavně ho používáme k uložení metadat do PDF (včetně obsahu).
%% POZOR, nezapomeňte vyplnit jméno práce a autora.
\usepackage[ps2pdf,unicode]{hyperref}   % Musí být za všemi ostatními balíčky
\hypersetup{pdftitle=Název práce}
\hypersetup{pdfauthor=Jméno Příjmení}

%%% Drobné úpravy stylu

% Tato makra přesvědčují mírně ošklivým trikem LaTeX, aby hlavičky kapitol
% sázel příčetněji a nevynechával nad nimi spoustu místa. Směle ignorujte.
\makeatletter
\def\@makechapterhead#1{
  {\parindent \z@ \raggedright \normalfont
   \Huge\bfseries \thechapter. #1
   \par\nobreak
   \vskip 20\p@
}}
\def\@makeschapterhead#1{
  {\parindent \z@ \raggedright \normalfont
   \Huge\bfseries #1
   \par\nobreak
   \vskip 20\p@
}}
\makeatother

% Toto makro definuje kapitolu, která není očíslovaná, ale je uvedena v obsahu.
\def\chapwithtoc#1{
\chapter*{#1}
\addcontentsline{toc}{chapter}{#1}
}

\begin{document}

% Trochu volnější nastavení dělení slov, než je default.
\lefthyphenmin=2
\righthyphenmin=2

%%% Titulní strana práce

\pagestyle{empty}
\begin{center}

\large

Univerzita Karlova v Praze

\medskip

Matematicko-fyzikální fakulta

\vfill

{\bf\Large DIPLOMOVÁ PRÁCE}

\vfill

\centerline{\mbox{\includegraphics[width=60mm]{../img/logo.eps}}}

\vfill
\vspace{5mm}

{\LARGE Jméno a příjmení autora}

\vspace{15mm}

% Název práce přesně podle zadání
{\LARGE\bfseries Název práce}

\vfill

% Název katedry nebo ústavu, kde byla práce oficiálně zadána
% (dle Organizační struktury MFF UK)
Název katedry nebo ústavu

\vfill

\begin{tabular}{rl}

Vedoucí diplomové práce: & Jméno a příjmení s~tituly \\
\noalign{\vspace{2mm}}
Studijní program: & program \\
\noalign{\vspace{2mm}}
Studijní obor: & obor \\
\end{tabular}

\vfill

% Zde doplňte rok
Praha ROK

\end{center}

\newpage

%%% Následuje vevázaný list -- kopie podepsaného "Zadání diplomové práce".
%%% Toto zadání NENÍ součástí elektronické verze práce, nescanovat.

%%% Na tomto místě mohou být napsána případná poděkování (vedoucímu práce,
%%% konzultantovi, tomu, kdo zapůjčil software, literaturu apod.)

\openright

\noindent
Poděkování.

\newpage

%%% Strana s čestným prohlášením k diplomové práci

\vglue 0pt plus 1fill

\noindent
Prohlašuji, že jsem tuto diplomovou práci vypracoval(a) samostatně a výhradně
s~použitím citovaných pramenů, literatury a dalších odborných zdrojů.

\medskip\noindent
Beru na~vědomí, že se na moji práci vztahují práva a povinnosti vyplývající
ze zákona č. 121/2000 Sb., autorského zákona v~platném znění, zejména skutečnost,
že Univerzita Karlova v Praze má právo na~uzavření licenční smlouvy o~užití této
práce jako školního díla podle §60 odst. 1 autorského zákona.

\vspace{10mm}

\hbox{\hbox to 0.5\hsize{%
V ........ dne ............
\hss}\hbox to 0.5\hsize{%
Podpis autora
\hss}}

\vspace{20mm}
\newpage

%%% Povinná informační strana diplomové práce

\vbox to 0.5\vsize{
\setlength\parindent{0mm}
\setlength\parskip{5mm}

Název práce:
Název práce
% přesně dle zadání

Autor:
Jméno a příjmení autora

Katedra:  % Případně Ústav:
Název katedry či ústavu, kde byla práce oficiálně zadána
% dle Organizační struktury MFF UK

Vedoucí diplomové práce:
Jméno a příjmení s tituly, pracoviště
% dle Organizační struktury MFF UK, případně plný název pracoviště mimo MFF UK

Abstrakt:
% abstrakt v rozsahu 80-200 slov; nejedná se však o opis zadání diplomové práce

Klíčová slova:
% 3 až 5 klíčových slov

\vss}\nobreak\vbox to 0.49\vsize{
\setlength\parindent{0mm}
\setlength\parskip{5mm}

Title:
% přesný překlad názvu práce v angličtině

Author:
Jméno a příjmení autora

Department:
Název katedry či ústavu, kde byla práce oficiálně zadána
% dle Organizační struktury MFF UK v angličtině

Supervisor:
Jméno a příjmení s tituly, pracoviště
% dle Organizační struktury MFF UK, případně plný název pracoviště
% mimo MFF UK v angličtině

Abstract:
% abstrakt v rozsahu 80-200 slov v angličtině; nejedná se však o překlad
% zadání diplomové práce

Keywords:
% 3 až 5 klíčových slov v angličtině

\vss}

\newpage

%%% Strana s automaticky generovaným obsahem diplomové práce. U matematických
%%% prací je přípustné, aby seznam tabulek a zkratek, existují-li, byl umístěn
%%% na začátku práce, místo na jejím konci.

\openright
\pagestyle{plain}
\setcounter{page}{1}
\tableofcontents

%%% Jednotlivé kapitoly práce jsou pro přehlednost uloženy v samostatných souborech
\chapter*{Úvod}
\addcontentsline{toc}{chapter}{Úvod}

Následuje několik ukázkových kapitol, které doporučují, jak by se
měla diplomová práce sázet. Primárně popisují použití \TeX{}ové
šablony, ale obecné rady poslouží dobře i~uživatelům jiných
systémů.

\chapter{Název první kapitoly}

\section{Název první podkapitoly v první kapitole}

\section{Název druhé podkapitoly v první kapitole}


\chapter{Název druhé kapitoly}

\section{Název první podkapitoly v druhé kapitole}

\section{Název druhé podkapitoly v druhé kapitole}



% Ukázka použití některých konstrukcí LateXu (odkomentujte, chcete-li)
% %%% Ukázka použití některých konstrukcí LaTeXu

\subsection{Ukázka \LaTeX{}u}
\label{ssec:ukazka}

V~této krátké části ukážeme použití několika základních konstrukcí \LaTeX{}u,
které by se vám mohly při psaní práce hodit.

Třeba odrážky:

\begin{itemize}
\item Logo Matfyzu vidíme na obrázku~\ref{fig:mff}.
\item Tato subsekce má číslo~\ref{ssec:ukazka}.
\item Odkaz na literaturu~\cite{lamport94}.
\end{itemize}

Druhy pomlček:
červeno-černý (krátká),
strana 16--22 (střední),
$45-44$ (minus),
a~toto je --- jak se asi dalo čekat --- vložená věta ohraničená dlouhými pomlčkami.
(Všimněte si, že jsme za \verb|a| napsali vlnovku místo mezery: to aby se
tam nemohl rozdělit řádek.)

% Makro na české uvozovky (novější verze LaTeXu ho už mají zabudované)
\newcommand{\uv}[1]{\quotedblbase #1\textquotedblleft}
\uv{České uvozovky.}

\newtheorem{theorem}{Věta}
\newtheorem*{define}{Definice}	% Definice nečíslujeme, proto "*"

\begin{define}
{\sl Strom} je souvislý graf bez kružnic.
\end{define}

\begin{theorem}
Tato věta neplatí.
\end{theorem}

\begin{proof}
Neplatné věty nemají důkaz.
\end{proof}

\begin{figure}
	\centering
	\includegraphics[width=30mm]{../img/logo.eps}
	\caption{Logo MFF UK}
	\label{fig:mff}
\end{figure}


\chapter*{Závěr}

Stručně uveďte, co nového vaše DP přináší. Zmiňte jen ty nejdůležitější výsledky. Jedná se o souhrn výsledků. Neopakujte tedy fakta z Materiálu a metodiky.  Po přečtení souhrnu by mělo být každému zcela jasné, o čem vaše DP pojednává, aniž by četl jakýkoliv jiný text. Je slušné se vejít na jednu stránku. 

\addcontentsline{toc}{chapter}{Závěr}


%%% Seznam použité literatury
%%% Seznam použité literatury je zpracován podle platných standardů. Povinnou citační
%%% normou pro diplomovou práci je ISO 690. Jména časopisů lze uvádět zkráceně, ale jen
%%% v kodifikované podobě. Všechny použité zdroje a prameny musí být řádně citovány.

\def\bibname{Seznam použité literatury}
\begin{thebibliography}{xxxxx--99}
\addcontentsline{toc}{chapter}{\bibname}

\bibitem[Lamport--94]{lamport94}
  {\sc Lamport,} Leslie.
  \emph{\LaTeX: A Document Preparation System}.
  2. vydání.
  Massachusetts: Addison Wesley, 1994.
  ISBN 0-201-52983-1.

\end{thebibliography}


%%% Tabulky v diplomové práci, existují-li.
\chapwithtoc{Seznam tabulek}

%%% Použité zkratky v diplomové práci, existují-li, včetně jejich vysvětlení.
\chapwithtoc{Seznam použitých zkratek}

%%% Přílohy k diplomové práci, existují-li (různé dodatky jako výpisy programů,
%%% diagramy apod.). Každá příloha musí být alespoň jednou odkazována z vlastního
%%% textu práce. Přílohy se číslují.
\chapwithtoc{Přílohy}

\openright
\end{document}
