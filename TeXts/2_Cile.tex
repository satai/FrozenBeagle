\chapter{Cíle}

%Přehledným způsobem se  uvede seznam  cílů DP a stručně se zdůvodní pracovní postupy navrhované k jejich řešení

Cíle této práce jsou:


\begin{enumerate}
    \item{Navrhnout a vyvinout program nebo programy, které mohou simulovat některé evoluční procesy, zejména přizpůsobení
    se populací změnám podmínek prostředí v závislosti na podílu pleiotropních alel a na podílu alel s frekvenčně
    závislou selekcí, konkrétně těch, u~kterých je opačný směr působení na fenotyp, pokud jsou v lokusu v jedné nebo
    dvou instancích (dále jen negativně dominantní alely).
    \item{Použít tento program nebo programy k otestování hypotézy, že populace se studovanými typy alel  pro zvolenou
         sadu parametrů častěji vymírají.}
        \begin{itemize}
            \item{Populace, kde je nenulové zastoupení negativně dominantních alel nebo alel ovlivňujících více
                  složek fenotypu, častěji vymírají.}
            \item{Tento efekt negativně dominantních alel roste s množstvím alel ovlivňujících více složek fenotypu.}
        \end{itemize}
    }
    \item{Použít tento program nebo programy k otestování hypotézy, že se populace se studovanými typy alel pro zvolenou
         sadu parametrů hůře přizpůsobují změnám podmínek.
        \begin{itemize}
            \item{Populace, kde je nenulové zastoupení negativně dominantních alel a zároveň dostatečné množství
                       alel ovlivňujících více složek fenotypu, dosahují menších průměrů i percentilu fitness (t.j.
                       nedostanou se tak blízko optimu).}
            \item{Tento efekt negativně dominantních alel roste s množstvím alel ovlivňujících více složek fenotypu.}
        \end{itemize}
    }
    \item{Použít tento program nebo programy k otestování hypotézy, že se populace se studovanými typy alel pro zvolenou
         sadu parametrů při změnách podmínek pomaleji přizpůsobují a naopak později jejich fitness rostou rychleji.

        \begin{itemize}
        \item{Populace, kde je nenulové zastoupení negativně dominantních a zároveň dostatečné množství alel ovlivňujících
              více složek fenotypu, mají po změně prostředí větší směrnici vyjadřující růst fitness,
              na konci období s novým stavem prostředí pak menší směrnici růstu fitness.}
        \item{Tento efekt je nastává i po návratu prosředí do původního stavu a je silnější.}
        \item{Tento efekt negativně dominantních alel roste s množstvím alel ovlivňujících více složek fenotypu.}
        \end{itemize}
    }
\end{enumerate}