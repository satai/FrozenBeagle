\section{Primární data}
\label{sec:primarydata}

Primární data ze simulací jsou příliš rozsáhlá, aby bylo možné je zařadit jako textovou přílohu. Jsou přiložena elektronicky k práci v SISu.

Zde je stručně popsán jejich formát a jak je v načíst pro případnou další analýzu.

Data jsou uložena v souborech \texttt{result\_\textit{XXX}\_0.\textit{YY}\_0.\textit{ZZ}}, kde
\textit{XXX} je počáteční velikost populace,
0.\textit{YY} je podíl alel, u kterých se projevuje negativní dominance,
a 0.\textit{ZZ} je podíl pleiotropních alel.

Každý soubor obsahuje statistiky pro všech 512 běhů \textit{fbeagle} s danými parametry. Ty jsou uloženy jako
haskellový literál typu \texttt{[[(String, [(Integer, Double)])]]}, ale vzhledem k jednoduché textové
reprezentaci je možné je připadně naparsovat i v jiném programovacím jazyce.

Semantika dat je následující -- jde o seznam výsledků pro všechny běhy s danou dvojicí parametrů. Každý výsledek
je seznam dvojic, prvním členem je jméno měřené veličiny (např. \textit{"avg\_fitness"} pro průměrnou hodnotu
fitness v populaci v daném kroku) a druhým členem seznam hodnot. Každá hodnota je dvojice, kde první člen je
číslo kroku simulace a druhý člen je hodnota dané veličiny. Tato hodnota je vždy číslo s plovoucí desetinnou čárkou,
i pokud by charakter veličiny umožňoval užít celé číslo. Pokud hodnota neexistuje (např. je to průměrná fitness vyhynulé
populace), je hodnotou \textit{NaN} (Not A Number).

Pro případné načtení dat z \textit{REPLu Haskellu} lze užít napřiklad tento kód:

\begin{code}[commandchars=\\\{\}]
handle <- openFile (fileName) ReadMode
contents <- hGetContents handle
\textbf{let}
        datas = read contents :: [[(String, [(Integer, Double)])]]
\end{code}
