\chapter*{Úvod}
\addcontentsline{toc}{chapter}{Úvod}

%Úvod by měl být literárním přehledem, který uvede i méně zasvěceného čtenáře do dané problematiky. To ovšem neznamená, že Úvod musí být kolosální dílo na mnoho stran s vyčerpávajícím seznamem prací, které se o studovaném taxonu a problematice objevily po dobu trvání lidské historie. Snažte se spíš z literatury vytáhnout podstatné informace a ty dát do souvislostí. Musíte předpokládat, že čtenář je ignorant, který o vámi řešené problematice neví téměř nic. Začněte nejlépe půlstranou zcela
%obecného úvodu (sice pro biology, ale v oboru DP neodborníky), kde nastíníte, o čem vlastně celý Úvod bude. Ideální je, pokud dříve nepoučený čtenář na základě této půlstrany a vašich cílů na konci Úvodu pochopí, o čem by vaše DP měla být. Studovaná problematika by měla být vztažena i k nějakým zásadním otázkám širšího rázu. Prostudujte tedy i jiné práce než o vašem oblíbeném taxonu.
%Příklad: Pokud studuji soužití dvou druhů žížal na jedné lokalitě, měl bych kromě prací o žížalách také zmínit obecnější práce třeba o koexistenci druhů a zauvažovat, jak moje žížaly přispějí k objasnění nevyřešených záhad této problematiky.
%
%Chybou je opomenout zásadní práce, které se tématu věnovaly. Pokud však někoho citujete, musí být pro to důvod.
%
%Příklad: Nepište  žížaly studoval také Neděla (2003) a Sonntag (2001)  napište raději třeba Neděla (2000) zjistil, že žížala je celkem dlouhá.
%
%Pozor na správné použití závorek.
%
%Příklad: Neděla  Sonntag  (2000) zjistili, že žížala je celkem dlouhá. ALE: žížala je celkem dlouhá (Neděla  Sonntag 2000). Pokud je autorů článku více než dva, obvykle uvedeme prvního autora a ostatní se schovají pod zkratku et al.  Například Sonntag et al. (2004).
%
%Vyhledáním fakt z literatury vaše práce nekončí. Fakta je třeba kriticky zhodnotit a jasně ukázat, kde jsou zde ještě mezery našeho poznání. Nezapomeňte však jasně rozlišovat mezi fakty, názory a spekulacemi.
%
%Příklad: Neděla (2000) sice tvrdí, že žížala je celkem dlouhá, ale Sonntag (2001) uvádí, že by žížala mohla být i celkem krátká. Precizní měření žížaly však nebylo dosud provedeno.
%
%Jasně napište, co nového by vaše DP mohla přinést a díky čemu jste toho dosáhl (dosáhla).
%
%Příklad: Vyvinutí nového měřítka otvírá nové možnosti v zjišťování délky žížal. Vznesené otázky o délce žížaly mohou být tedy konečně zodpovězeny.
%
%Úvod zakončete uvedením cílů své práce, nejlépe ve formě testovatelných hypotéz.
%
%Příklad: Cílem mé práce je změřit žížalu pomocí přesného měřítka. Chtěl bych dokázat, že žížala je delší než deset centimetrů.
%
%Hypotézy (či otázky) by měly jasně vyplývat z Úvodu. Neměla by se tedy zde objevit téma, které není v Úvodu vůbec zmíněno.
%
%Příklad: Celý Úvod je jen o měření a délce žížal. Neměla by se tedy nakonec nenadále objevit hypotéza o rozmnožování, byť by se jednalo o rozmnožování žížal.
% Celý text musí být logicky provázaný. Věty na sebe musí navazovat. Vyprávíte příběh a jeho pointou jsou vaše cíle a výsledky.

Od publikování Darwinova \textit{O vzniku druhů} uplynulo více než jeden a půl století a za tuto dobu byly jednotlivé části
jeho hypotézy mnoha způsoby komentovány, kritizovány i doplňovány.

Mnoho z původních Darwinových názorů se nestalo posledním slovem v diskusi. Řada sporů se točila kolem významu role
přírodního výběru, možností a významu dědění vlastností získaných za života organismu, možnosti vzniku některých
adaptivních struktur i kolem základní otázky, zda jsou nejdůležitějším subjektem evoluce jedinci\ldots

Stranou kritiky nezůstala ani běžná interpretace Darwinových představ o rychlosti evoluce nebo přesněji o jejím rytmu.
Podle ní si Darwin a částečně i Wallace, v mnohém pod vlivem předchozích stanovisek Charlese Lyella,
představovali rychlost evoluce jako nepříliš proměnlivou. V takovém podání byla evoluce dlouhá řada pozvolných změn
v~neměnném tempu. To dnes nazýváme \textit{gradualistickým} stanoviskem.

Alternativní stanovisko má také starší kořeny, než jen jako reakce na Darwinovy teorie. Ponecháme-li stranou například
katastrofismus, tak i hypotéza, že vývoj neprobíhá plynule, našla mnoho zastánců. Toto přesvědčení dnes
obvykle nazýváme \textit{punktuacionalismem}.

Proč by rychlost evoluce neměla být plynulá, postulovala řada hypotéz. Jednou z nich se stala teorie zamrzlé plasticity
profesora Jaroslava Flegra publikovaná na sklonku minulého století. Ta tvrdí, že dědivost fenotypu u pohlavně se
rozmnožujících druhů je nízká, a proto nejsou tyto druhy schopné většího přizpůsobení se změnám prostředí. Druh se vůči
těmto změnám chová elasticky -- po skončení selekčního tlaku se genotyp a tím i fenotyp druhu vrací do původního stavu.
Příčinou jsou mezialelické interakce. Rozmrznutí -- přechod druhu do plastického stadia -- je možný jen za specifických
podmínek, například při peripatrické speciaci nebo při průchodu hrdlem lahve, kdy se drasticky sníží genetický
polymorfismus.

K jejím častým kritikám patří to, že zůstala v podobě slovního popisu. Tato práce si klade za cíl formalizovat alespoň
část vlastností druhu, které by měly mít vliv na jeho plasticitu, a ověřit prostřednictvím počítačové simulace, zda
tyto vlastnosti mají vliv. Zejména těmi vlastnostmi bude množství alel ovlivňujících více fenotypových vlastností a
množství alel, u kterých jejich dominantní projev působí na fenotyp opačně, než pokud se v lokusu vyskytnou jen v jedné
instanci.
