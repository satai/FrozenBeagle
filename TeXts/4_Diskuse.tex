\chapter{Diskuse}
%
%Diskusi nepodceňujte. Je to nejdůležitější část a podle toho taky nejsložitější na napsání. Může se stát, že po napsání Diskuse budete muset přepsat třeba celý Úvod. Napsání kvalitní Diskuse vyžaduje spoustu času. Počítejte s tím. Za jeden večer to nenapíšete.
%
%Citát: \uv{Z krátké diskuse čiší stupidita autora, který vlastně neví co diskutovat.} Dr. B. Mandák, botanický ústav AVČR
%
%Bývá dobré začít shrnutím a interpretací vašich výsledků. Diskuse ale musí také ukázat jak výsledky zapadají do toho, co je o dané problematice známo. Musíte oddiskutovat jak soulad získaných výsledků s publikovanými tak i jejich nesoulad. Domníváte-li se, že jsou vaše výsledky zcela nové, pak vysvětlete v čem je jejich originalita. Pokud je nesoulad mezi výsledky vašimi a jiných badatelů, pak je nutné vysvětlit, čím k tomu mohlo dojít.
%
%Příklad: Sonntag (2001) sice uvádí, že by žížala mohla být i celkem krátká, neměl však k dispozici přesné měřítko, kterým žížalu měřím já.
%
%Nezapomeňte, že jste si v Úvodu vytyčili otázky a cíle. Na ně je třeba reagovat. Můžete i naznačit, jakým směrem by se teď po vašich zásadních objevech měl ubírat další výzkum. Můžete i formulovat nové hypotézy, které by měly být v budoucnu testovány. Pořadí diskutovaných okruhů by mělo být stejné jako bylo v Úvodu. Čtenář DP by měl být na Diskusi připraven Úvodem. Nemělo by se tedy stát, že se v Diskusi zjeví překvapivá fakta z prací jiných autorů, o nichž není v Úvodu ani zmínky.

\section{Název první podkapitoly v druhé kapitole}

\section{Název druhé podkapitoly v druhé kapitole}

