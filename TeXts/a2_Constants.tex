\section{Konstanty}
\label{sec:constants}



\begin{table}[h]
\centering
\label{constants}
\begin{tabularx}{\textwidth}{| l | l | l|}
\hline
\texttt{dimensionCount}             & 4      & Počet dimenzí prostoru fenotypů. \\
\hline
\texttt{optimumChangeGeneration}    & 8192   & \makecell[l]{Počet kroků jednoho úseku\\simulace s konstatním prostředím}\\
\hline
\texttt{accidentDeathProbability}   & 0.0    & \makecell[l]{Pravděpodobnost náhodného úmrtí\\jedince v jednom kroku simulace.\\$k_4$ v turbidostatické regulaci.} \\
\hline
\texttt{probabilityAlleleMutation}  & 0.0002 & \makecell[l]{Pravděpodobnost, že v jednom\\kroku simulace dojde k mutaci\\jedné konkrétní alely.}\\
\hline
\texttt{optimumSizeCoefficient}     & 12.0   & \makecell[l]{Násobek normálního rozdělení, o\\který se při změnách optima\\posune každá z dimenzí.}  \\
\hline
\texttt{fitnessDecreaseCoefficient} & -0.005 & \makecell[l]{Určuje, jak rychle klesá s rostoucí\\vzdáleností od optima fitness\\viz \citet{tenaillon2014utility}}\\
\hline
\texttt{negativeDominanceScale}     & 1.5    & \makecell[l]{Modus rozdělení poměru velikosti\\příspěvku alely u homozygota\\k jejímu příspěvku, pokud se\\vyskytuje jednou} \\
\hline
\texttt{negativeDominanceShape}     & 1.5    & \makecell[l]{$\alpha$ parametr Paretova rozdělení\\ poměru velikosti příspěvku alely\\ u homozygota k jejímu příspěvku,\\ pokud se vyskytuje jednou }\\
\hline
\end{tabularx}
\caption{Konstanty nastavené pro všechny simulace}
\end{table}
