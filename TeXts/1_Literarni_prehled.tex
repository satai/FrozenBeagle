\chapter{Literární přehled}


XXX

%\section{Gradualismums vs.}
%
\section{Je to jinak. Darwin}
%
%\section{Historie}
%
%\subsection{Kvantová evoluce}
%
\section{Zamrzlá plasticita}

\subsection{Problém evoluce pohlavně se rozmnožujících druhů}

První verze teorie zamrzlé plasticity byla publikována jako \citet{flegr1998origin}. Článek upozorňuje na to, že
klasický Darwinův model přirozeného výběru může dobře fungovat u asexuálních druhů, kde je dědivost znaků téměř
dokonalá.
Pokud při rozmnožování dojde k rekombinaci, tak je situace jiná. Přestože alely se mezi generacemi přenášejí
nepozměněné, tak zejména kvantitativní vlastnosti jsou kódovány více geny s malým efektem, navíc zapojenými do
složité sítě vazeb. Krom toho působení alel nemusí být nutně aditivní, jejich projev závisí na kontextu,
ve kterém se vyskytují.

To vede k tomu, že se kvantitativní vlastnosti se dědění průměrují a změny mají tendenci po
několika generacích vymizet. Navíc se nové výhodné mutace v potomcích objeví v z poloviny odlišném kontextu. To také
omezuje možnost jejich fixace -- je možné, že nový kontext způsobí odlišnou expresi takové alely nebo vlastnost bude
v kontextu jiných vlastností spíše škodlivá a výsledný vliv na fenotyp nebude tak výhodný jako u rodiče. Navíc
úspěšnost alely může záviset na její frekvenci v populaci. Z nízké dědivosti znaků plyne i nźká dědivost fitness.

U pohlavně se rozmnožujících druhů by tak byly dobře dědivé a tedy selekcí fixovatelné hlavně alely s nerealisticky
velkým účinkem na fitness, navíc s aditivním působením.

Podobně je problematické i Dawkinsem zpopularizované řešení tohoto problému v podobě teorie sobeckého genu.
Podle ní je pro alelu klíčové ne zlepšit fenotyp jedince, ale uspět v soupeření s jinými alelami téhož genu
(k čemuž jí zlepšení fenotypu může ale nemusí pomoci). Problém s dědivostí však zůstává, pokud není působení
alely pouze aditivní a je například ovlivněno epistatickými jevy.

Na základě těchto úvah profesor Flegr navrhnul pro pohlavně se rozmnožující druhy odlišný model evoluce -- zamrzlou
plasticitu.

\subsection{Teorie zamrzlé plasticity}

Řešením může být najít sadu podmínek, kdy je dědivost vysoká. Taková situace se může vyskytnout, pokud je genetická
rozmanitost populace nízká. Pak se může plně projevit působení přirozeného výběru a vytvořit a případně zafixovat i
výraznější a ireverzibilní změny. Později se vlivem mutací pomorfismus opět zvýší.

Navrhnuty jsou tři mechanismy pro vznik takové populace -- kolonizace nové oblasti, efekt průchodu hrdlem lahve a
systematický asortativní výběr partnera. V takových případech se může výrazně snížit genetický polymorfismus. Alely
se v následujících generacích objevují v podobném kontextu jako v předchozích. Navíc, pokud je nízká efektivní
velikost populace dlouhodobější, může se výrazněji projevit genetický drift a dále snižovat polymorfismus. Je
pravděpodobné, že druh v takové situaci vyhyne, ale pokud přežije a počet jedinců vzroste, vznikne početná geneticky
homogenní populace. V té může účině působit přirozený výběr\footnote{
A nejen přirozený výběr, ale i výběr umělý. Proto také rozdíly ve stabilitě fenotypu mezi samosprašnými a cizosprašnými
kulturními rostlinami.\citep[str. 267--269]{flegr2002lysenko}
}.

Druh je tedy plastický obvykle jen na počátku své existence. V té době se u něj mohou objevit anagenetické změny.
Později se naakumuluje genetický polymorfismus a druh zamrzá -- stane se elastickým. V této fázi je možnost jeho
odpovědi na selekční tlaky omezená a změny často odezní po omezení takového tlaku. Reakce na změny je sice omezená
co do rozsahu, ale může být relativně rychlá, protože proběhne změnou frekvence alel do nového rovnovážného stavu
\citep[str. 194]{flegr2016}.

Zamrzání lze chápat i jako výhodý proces, protože zabraňuje druhu oportunisticky příliš výrazně reagovat na
krátkodobé změny, ale udržuje u druh stabilní přizpůsobení dlouhodobým průměrům prostředí \citep[str. 195]{flegr2016}.
Jde tedy o jedno z mnoha možných vysvětlení výhodnosti sexuálního rozmnožování. Důsledkem by podle
\citet[str. 166--167]{toman2015} mělo být, že pohlavní druhy preferují proměnlivější, heterogenější
a strukturovanější prostředí a asexuální druhy prostředí stabilnější.

\subsection{Teorie zamrzlé evoluce}

Teorie zamrzlé plasticity a zamrzlé evoluce nebyly dříve (třebas v popularizační knize \citet{flegr2006})
jasně rozlišeny, ale dnes je jde o dvě zřetelně oddělené hypotézy.
Teorie zamrzlé evoluce má mezi svými předpoklady i teorii zamrzlé plasticity, zaměřuje se však na makroevoluční
hledisko. Některé znaky se podle ní stávají ireverzibilně elastické a postupně se hromadí mechanismem třídění
na základě stability \citep[str. 294--296]{flegr2016}.

\section{Další punktcionalistické teorie}

%\subsection{Reakce}
%
\section{Srovnání hypotéz}



\section{Možnosti testování}
%\subsection{Historie}
%\subsection{Mechanismy}
%\subsection{Pokusy o otestování}
%
%\section{Simulace}
%\subsection{Optimalizace}
%\subsection{Simulace}
