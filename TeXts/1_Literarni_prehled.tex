\chapter{Literární přehled}


XXX

%\section{Gradualismums vs.}
%
\section{Darwin}

\section{Předchozí punktuacionalistické teorie}

Přehled punktuacionalistických teorií předcházející teorii zamrzlé plasticity samozřejmě nemůže být úplný, své místo
by si v obecnějším textu jistě zasloužil například G. G. Simpson a jeho teorie kvantové evoluce, ale jeho snaha o
vysvětlení se týkala spíše makroevoluce a vzniku větších kladů \citep[str. 529]{gould2002structure}. XXX

\subsection{Teorie posunující se rovnováhy}

\citet[str. 104--105]{wright1931} upozornil na základě starších výsledků pokusů na octomilkách na to, že pro některé
geny existuje více alel\footnote{Používal dnes již zastaralý termín \textit{allelomorph.}} než dvě a že by nebylo
překvapivé, kdyby to byla běžná situace pro mnoho genů.

Na tento postřeh navázal ve \citet[str. 356]{wright1932}, kde ukázal, že v takovém případě je počet možných kombinací
alel již pro několik lokusů astronomický, o mnoho řádů větší, než je počet jedinců druhu. Drtivá většina kombinací
tedy nikdy nevznikne.

Pokud by nekonečně početný druh, kde se jedinci mohou libovolně křížit, žil za neměnných podmínek, tak by se po čase
frekvence alel měly ustálit (str. 360). V terminologii adaptivních krajin, kterou Wright také používá, druh
zabere nějakou její část. Obsazená plocha je stálá, ale žádní jedinci nejsou stejní.

K posunu může dojít, pokud v této situaci vznikne nová výhodná mutace\footnote{Což také změní rozložení adaptivní
krajiny, například zvýšením některého jejího vrcholku.}. Wright argumentuje, že evoluce probíhající prostřednictvím
tohoto mechanismu bude velmi pomalá, protože se takové mutace budou objevovat zřídka a čas pro dostatečné zvýšení
frekvence nových alel bude obrovský.

Wright navrhuje tři mechanismy, které mohou způsobit rychlejší evoluci -- vyšší tempo mutací, větší intenzitu
selekce a změnu velikosti nebo struktury populace.

Pokud se velikost populace zmenší pod určitou úroveň, sníží se její genetický polymorfismus, zmenší
okupované území v adaptivní krajině a zvýší se význam náhodných jevů. Druh se může vydat na náhodnou procházku
adaptivní krajinou. Ta ho může zavést i do údolí, čemuž by za jiných okolností bránila selekce nebo skrz
údolí přejít na jiný vrchol (str. 362). Takový postup však také nemusí být příliš rychlý (str. 363).

Proto se Wrightovi jeví jako zajímavý případ, kdy je početnější druh rozdělen do mnoha menších místních demů \footnote{
Wright podobně jako Darwin užívá slova \textit{race}.
}. Takové demy se mohou adaptivní krajinou pohybovat tak jako malé druhy z předchozího odstavce, navíc jsou tím, že jsou
v jejích různých částech, vystaveny i různým směrům selekčního působení. K tomu může mezi těmito demy zapůsobit
skupinová selekce. Pokud bude izolace skupin úplná, mělo by dojít k rychlé speciaci.

Z charakteru náhodné procházky je patrné, že ne všechny podobné události jsou adaptivní.\footnote{
    Musím se přiznat, že jsem se při čtení Wrightových článků neubránil obdivu, jak daleko šlo v úvahách
    zajít dvacet let před centrálním dogmatem.
}

%Evoluce je podle Wrighta (str. 365) závislá na vztahu mutace, selekce a charakteru křížení mezi jedinci. Každý z těchto
%procesů musí probíhat s přiměřenou intenzitou. Specialně pro sexuálně se rozmnožující druhy s nízkou mírou mutace
%a umírněnou selekcí hraje velkou roli náhoda -- může vzniknout jen zlomek možných kombinací alel. Ke speciacím dochází
%

\subsection{Teorie genetické revoluce}

Podle \citep[str. 140]{mayr1989} byla hlavním přínosem jeho teorie tvrzení, že nejrychlejší změny se neobjevují u
početných a rozšířených druhů, ale naopak v malých zakladatelských populacích.

Již ve \citep{mayr1954change} hned v první větě zdůrazňuje význam izolace jako důležitého evolučního faktoru a klade ji
na stejnou úroveň, jako je mutace, rekombinace a selekce.

Mayr vyšel z pozorování papuánských ledňáčků rodu \textit{Tanysiptera}, u kterých jsou pozoruhodně rozložené druhy.
Na geograficky rozlehlé Nové Guinei, kde jsou podstatné rozdíly v místních podmínkách,  jsou místní populace
\textit{T. galatea} překvapivě podobné, prakticky není možné rozlišit jedince z různých konců ostrova, přestože
žijí ve značně odlišných prostředích. Naopak populace na okolních ostrovech se podstaně liší, a to i pokud jsou jejich
životní podmínky podobné.

To naznačuje, že selekce nehrála rozhodující roli. Stejně tak nelze jako vysvětlení přijmout drift, protože populace
jsou příliš početné.

Protože se nejedná o osamělou výjimku, ale podobných případů homogenních populací na velkých a často heterogenních
územích a odlišných populací v oddělených oblastech je mnoho (str. 159), volá situace po vysvětlení.



\subsection{Founder-flush model}

\subsection{Genetic transilience model}

\subsection{Přerušované rovnováhy}

Teorie přerušovaných rovnováh navazuje na

\section{Zamrzlá plasticita}

\subsection{Problém evoluce pohlavně se rozmnožujících druhů}

První verze teorie zamrzlé plasticity byla publikována jako \citet{flegr1998origin}. Článek upozorňuje na to, že
klasický Darwinův model přirozeného výběru může dobře fungovat u asexuálních druhů, kde je dědivost znaků téměř
dokonalá.
Pokud při rozmnožování dojde k rekombinaci, tak je situace jiná. Přestože alely se mezi generacemi přenášejí
nepozměněné, tak zejména kvantitativní vlastnosti jsou kódovány více geny s malým efektem, navíc zapojenými do
složité sítě vazeb. Krom toho působení alel nemusí být nutně aditivní, jejich projev závisí na kontextu,
ve kterém se vyskytují.

To vede k tomu, že se kvantitativní vlastnosti se při dědění průměrují a změny mají tendenci po
několika generacích vymizet. Navíc se nové výhodné mutace v potomcích objeví v odlišném kontextu. To také
omezuje možnost jejich fixace -- je možné, že nový kontext způsobí odlišnou expresi takové alely nebo vlastnost bude
v kontextu jiných vlastností spíše škodlivá a výsledný vliv na fenotyp nebude tak výhodný jako u rodiče. Nadto
úspěšnost alely může záviset na její frekvenci v populaci. Z nízké dědivosti znaků plyne i nízká dědivost fitness.

U pohlavně se rozmnožujících druhů by tak byly dobře dědivé a tedy selekcí fixovatelné hlavně alely s nerealisticky
velkým účinkem na fitness, navíc s aditivním působením.

Podobně je problematické i Dawkinsem zpopularizované řešení tohoto problému v podobě teorie sobeckého genu.
Podle ní je pro alelu klíčové ne zlepšit fenotyp jedince, ale uspět v soupeření s jinými alelami téhož genu
(k čemuž jí zlepšení fenotypu může, ale nemusí pomoci). Problém s dědivostí však zůstává, pokud není působení
alely pouze aditivní a je například ovlivněno epistatickými jevy.\footnote{
Teorie vnitrolokusové mezialelické selekce pochází od W. D. Hamiltona, Colina Pittendrigha a George C. Williamse.
O tom, že si Dawkins byl problému závislosti vlivu i úspěchu alely na kontextu minimálně částečně vědom,
svědčí například \citet[str. 231]{dawkins2003}.
}

Na základě těchto úvah profesor Flegr navrhnul pro pohlavně se rozmnožující druhy odlišný model evoluce -- zamrzlou
plasticitu.

\subsection{Teorie zamrzlé plasticity}

Řešením může být najít sadu podmínek, kdy je dědivost vysoká. Taková situace se může vyskytnout, pokud je genetická
rozmanitost populace nízká. Pak se může plně projevit působení přirozeného výběru a vytvořit a případně zafixovat i
výraznější a ireverzibilní změny. Později se vlivem mutací polymorfismus opět zvýší.

Navrženy jsou tři mechanismy pro vznik takové populace -- kolonizace nové oblasti, efekt průchodu hrdlem lahve a
systematický asortativní výběr partnera. V takových případech se může výrazně snížit genetický polymorfismus. Alely
se v následujících generacích objevují v podobném kontextu jako v předchozích. Navíc, pokud je nízká efektivní
velikost populace dlouhodobější, může se výrazněji projevit genetický drift a dále snižovat polymorfismus. Je
pravděpodobné, že druh v takové situaci vyhyne, ale pokud přežije a počet jedinců vzroste, vznikne početná geneticky
homogenní populace. V té může účinně působit přirozený výběr.

Druh je tedy plastický obvykle jen na počátku své existence. V té době se u něj mohou objevit anagenetické změny.
Později se naakumuluje genetický polymorfismus a druh zamrzá -- stane se elastickým. V této fázi je možnost jeho
odpovědi na selekční tlaky omezená a změny často odezní po omezení takového tlaku. Reakce na změny je sice omezená
co do rozsahu, ale může být relativně rychlá, protože proběhne změnou frekvence alel do nového rovnovážného stavu
\citep[str. 194]{flegr2016}.

Zamrzání lze chápat i jako výhodný proces, protože zabraňuje druhu oportunisticky příliš výrazně reagovat na
krátkodobé změny, ale udržuje u něj stabilní přizpůsobení dlouhodobým průměrům prostředí \citep[str. 195]{flegr2016}.
Jde tedy o jedno z mnoha možných vysvětlení výhodnosti sexuálního rozmnožování. Důsledkem by podle
\citet[str. 166--167]{toman2015} mělo být, že pohlavní druhy preferují proměnlivější, heterogennější
a strukturovanější prostředí a asexuální druhy prostředí stabilnější.

\subsection{Teorie zamrzlé evoluce}

Teorie zamrzlé plasticity a zamrzlé evoluce nebyly dříve (třebas v popularizační knize \citet{flegr2006})
jasně rozlišeny, ale dnes   jde o dvě zřetelně oddělené hypotézy.\footnote{
Tato nejasnost je zřejmá například i v tabulce \ref{srovteorie} pocházející z \citet{flegr2010} a rozšířené v
\citet{flegr2013}.
}
Teorie zamrzlé evoluce má mezi svými předpoklady i teorii zamrzlé plasticity, zaměřuje se však na makroevoluční
hledisko. Znaky nerozmrzají stejně rychle, pro různé znaky je potřeba odstranit různé množství polymorfismu.
Některé znaky si tak udržijí polygenní stav i při speciacích, při kterých se jiné znaky změní na plastické.
Tím se stávají ireverzibilně elastické a postupně se hromadí mechanismem třídění
na základě stability \citep[str. 294--296]{flegr2016}.

Makroevolučně zamrzlé druhy jsou podle \citet[str. 2]{flegr2013} mikroevolučně elastické.

\section{Srovnání hypotéz}

\citet{flegr2013} přináší přehlednou tabulku, kde popsané teorie srovnává:

\footnotesize

\begin{longtable}{| l | l | l |}

\caption{Srovnání hypotéz}
\label{srovteorie}
\endfirsthead
\endhead

\hline
\textbf{Teorie}             & \textbf{Záměr}      & \textbf{Mechanismus} \\
\hline

\makecell[l]{Teorie\\ posunujících\\ se rovnováh} &
\makecell[l]{vysvětlení schopnosti \\ druhů s velkými \\ rozdělenými populacemi \\překonat údolí \\ v adaptivní krajině}  &
\makecell[l]{
1. fragmentace populace na malé subpopulace, \\
kde je nízká efektivita selekce \\
2. rozšíření a fixace nové alely (která je škodlivá, \\
je-li vzácná) v subpopulaci pomocí driftu \\
3. \uv{nakažení} dalších subpopulací jedinci s \\
novým genotypem vzniklým v úspěšné populaci \\
a vznik nových populací s takovýmito jedinci
}\\
\hline

\makecell[l]{Genetická \\ revoluce} &
\makecell[l]{vysvětlení role \\ efektu zakladatele \\ ve speciaci}   &
\makecell[l]{
1. změna v rovnovážné frekvenci alel v oddělených \\
subpopulacích  vzhledem k výběrovému efektu \\
2. selekce alel s nejlepším vlivem na fitness \\
namísto \uv{nejlépe spolupracujících} alel
}\\
\hline
\makecell[l]{Founder-flush} &
\makecell[l]{vysvětlení role \\ efektu zakladatele \\ ve speciaci}   &
\makecell[l]{
1. výběrový efekt vzhledem k jednorázovému \\
poklesu velikosti populace \\
2. růst populace v nové neobsazené ekologické \\
nice, kdy dochází k uvolnění všech podob selekce \\
a  umožnění přežití rekombinantů a mutantů se \\
suboptimálními fenotypy (překročení údolí v \\
adaptivní krajině)\\
3. dosažení (či přesažení) nosné kapacity lokality \\
a obnovení selekce
}\\
\hline

\makecell[l]{Genetická \\ transilience} &
\makecell[l]{vysvětlení role \\ efektu zakladatele \\ ve speciaci}   &
\makecell[l]{
1. výběrový efekt vzhledem k jednorázovému \\
poklesu velikosti populace či hybridizaci \\
2. nárůst míry selektovatelné genetické \\
variability díky změně neaditivní (a tedy \\
neselektovatelné) genetické variability na \\
aditivní a také vyšší pravděpodobností přežití \\
nositelů nových  mutací v rostoucí populaci, \\
což zvyšuje odpověď populace na selekci \\
3. obnovení omezení  velikosti populace a  selekce
}\\

\hline

\makecell[l]{Přerušované \\ rovnováhy} &
\makecell[l]{vysvětlení nespojité \\ povahy evoluce \\ a souběhu anageneze \\ a kladogeneze}   &
\makecell[l]{
rozličné mechanismy navržené Eldredgem a \\
Gouldem, včetně peripatrické speciace a účinné \\
selekce v neobvyklých podmínkách na periferii \\
areálu druhu, peripatrické speciace doprovázené \\
genetickou revolucí, třídění (dle Futuyma je \\
bez speciace jakákoli evoluční inovace vratná \\
vzhledem ke genovému toku), atd.
}\\
\hline

\makecell[l]{Zamrzlá \\ plasticita} &
\makecell[l]{
vysvětlení, proč \\
jsou staré druhy \\
mikroevolučně \\
elastické a \\
makroevolučně \\
zamrzlé, \\
jak se zamrzlé \\
druhy mohou \\
stát plastickými, \\
a postupně se \\
snižující míry \\
makroevoluce
}   &
\makecell[l]{
1. většina polymorfismu existujícího ve starých \\
druzích je zachována díky frekvenčně závislé \\
selekci tvořící propojenou síť odolávající změnám \\
frekvence alel \\
2. většina nových (potenciálně prospěšných) alel \\
je zachycena v této elastické síti díky pleiotropii \\
a jejímu vlivu na (stabilizované) frekvence starých \\
alel \\
3. v malých oddělených populacích nacházejících \\
se po několik generací na hraně vymření dojde \\
ke snížení síly selekce, včetně frekvenčně závislé, \\
a většina polymorfismu vymizí pomocí driftu \\
4. po nárůstu velikosti populace se již velká \\
a geneticky uniformní populace stane evolučně \\
plastickou -- v populaci bez sítí alel se nové \\
prospěšné mutace mohou šířit selekcí \\
5. znaky odolávající \uv{tání} se v genomu hromadí\\
tříděním na základě stability \\
6. dochází k hromadění trvale zamrzlých znaků \\
tříděním na základě stability v některých kladech \\
v průběhu makroevoluce
}\\
\hline
\hline

\multicolumn{3}{|l|}{
\makecell[l]{
Model genetické revoluce implicitně a model zamrzlé evoluce explicitně předpokládají, že \\
frekvenčně závislá selekce hraje zásadní roli ve stabilizaci genomu druhu. Makroevolučně \\
zamrzlé druhy jsou tedy podle těchto dvou teorií mikroevolučně elastické. \\
Podle teorií posunující se rovnováhy, vyčištění efektem zakladatele (founder-flush) a genetické \\
transilience jsou i mikroevolučně zamrzlé -– tedy mají významně sníženou plasticitu oproti \\
plastické fázi. Podle Futuymova modelu třídění mohou být makroevolučně zamrzlé druhy \\
mikroevolučně plastické. XXX cite
}
} \\

\hline
\end{longtable}
\normalsize

Z tabulky i předchozích popisů hypotéz je zřejmé, že se liší jak svým účelem, tak i výběrem procesů a mechanismů,
které podle nich hrají klíčovou roli.

Teorie obvykle operují s nějakou formou vzniku menší nebo strukturované populace jako mechanismem jejího přechodu
k rychlejší evoluci. Jednotlivé teorie často navrhují i další mechanismy (Wright rychlejší mutace nebo výrazný
selekční tlak, Templeton i hybridizaci aj.). Čím se často liší, je důvod, který způsobuje dlouhodobé zabrždění evoluce.


\section{Možnosti testování}

Možností, jak konfrontovat jednotlivé teorie, je několik, \citet[str. 4-5]{flegr2013} přináší řadu
předpovědí, kterými se liší gradualismus i jednotlivé skupiny punktuacionalistických teorií. Část těchto rozdílů se
týká makroevoluce, ale jsou zde i odlišnosti zajímavé z mikroevolučního hlediska.

Jedním z možných postupů, jak teorie otestovat, je zaměřit se na data o historickém průběhu evoluce, tak jak jsou nám
dnes k dispozici, případně srovnat známá chování současných druhů a skupin s jejich předpověďmi.

Paleontologická data použil \citet{mikulavs2008} pro pohled na průběh adaptací ramenonožců rodu \textit{Aegiromena} v
ordoviku. Jeho výsledky jsou v souladu s teorií zamrzlé plasticity -- postupně se snižovala schopnost druhů reagovat
na změny a zavádět novinky ve své morfologii.

Se současnými druhy  pracovala diplomová práce \citet{toman2013} později rozšířená a přepracovaná do \citet{toman2015}.
Tato metastudie se zaměřila na ekologické rozdíly mezi pohlavně se rozmnožujícími a nepohlavně se rozmnožujícími druhy.
Předpověď teorie zamrzlé plasticity, že nepohlavní skupiny budou spíše obývat abioticky i bioticky homogenní prostředí,
byla potvrzena. Naopak předpověď, že se v dlouhodobém měřítku zvládnou přizpůsobit širšímu rozpětí podmínek, konkrétně
teploty, v souladu s daty nebyla.

\subsection{Mechanismy}

Jinou linií pokusů může být snaha prověřit samotné navržené mechanismy, které způsobují stázi nebo naopak kterými
plasticita či elasticita druhu ovlivňují jeho úspěch. K tomu mohou složit nejen reálná data, ale i
modelování, například v podobě počítačových simulací.

\citet{Flegr139030} byl pokus porovnat chování dvou asexuálních druhů v proměnlivém prostředí. Plasticita jednoho a
elasticita druhého byla daná, mechanismus, který je zapříčinil, nebyl předmětem simulace.

To nechává prostor pro další práci v této oblasti, která ověří tvrzení teorie zamrzlé plasticity o tom, jak
různé druhy vlivu alel na fenotyp mají dopad na schopnost druhu reagovat na změny prostředí. Tím se prověří, zda
navržené mechanismy -- frekvenčně závislá selekce a pleiotropie -- skutečně mohou stát za vznikem elasticity a tím
vést k evoluční stázi.