\chapter{Diskuse}

%
%Diskusi nepodceňujte. Je to nejdůležitější část a podle toho taky nejsložitější na napsání. Může se stát, že po napsání Diskuse budete muset přepsat třeba celý Úvod. Napsání kvalitní Diskuse vyžaduje spoustu času. Počítejte s tím. Za jeden večer to nenapíšete.
%
%Citát: \uv{Z krátké diskuse čiší stupidita autora, který vlastně neví co diskutovat.} Dr. B. Mandák, botanický ústav AVČR
%
%Bývá dobré začít shrnutím a interpretací vašich výsledků. Diskuse ale musí také ukázat jak výsledky zapadají do toho, co je o dané problematice známo. Musíte oddiskutovat jak soulad získaných výsledků s publikovanými tak i jejich nesoulad. Domníváte-li se, že jsou vaše výsledky zcela nové, pak vysvětlete v čem je jejich originalita. Pokud je nesoulad mezi výsledky vašimi a jiných badatelů, pak je nutné vysvětlit, čím k tomu mohlo dojít.
%
%Příklad: Sonntag (2001) sice uvádí, že by žížala mohla být i celkem krátká, neměl však k dispozici přesné měřítko, kterým žížalu měřím já.
%
%Nezapomeňte, že jste si v Úvodu vytyčili otázky a cíle. Na ně je třeba reagovat. Můžete i naznačit, jakým směrem by se teď po vašich zásadních objevech měl ubírat další výzkum. Můžete i formulovat nové hypotézy, které by měly být v budoucnu testovány. Pořadí diskutovaných okruhů by mělo být stejné jako bylo v Úvodu. Čtenář DP by měl být na Diskusi připraven Úvodem. Nemělo by se tedy stát, že se v Diskusi zjeví překvapivá fakta z prací jiných autorů, o nichž není v Úvodu ani zmínky.

XXXXX


\section{Výsledky a jejich interpretace}

\section{Omezení}

\section{Vhodnost zvolených nástrojů pro simulaci}

Vzhledem k jisté neobvyklosti zvolených nástrojů, které byly vybrány pro implementaci simulace, zejména jazyka Haskell, je na místě stručně
vyhodnotit jejich adekvátnost.

V literatuře uváděné výhody kombinace funkcionálního programování a silných statických typových systémů (např. \citet{meijer2004static} pro
statické typování, \citet{hughes1989functional} pro FP a \citet{Chiusano2016} a \citet{ray2014large} pro kombinaci)
potvrdila i zkušenost při vývoji
\textit{fbeagle}. Přísná disciplína statického jazyka pomáhala s návrhem i často zabránila tomu, aby byl chybný kód vůbec přeložen\footnote{
Aby byla přísná pravidla ještě přísnější, tak jsem pro peklad výpočetní části \textit{fbeagle} používal přepínač \texttt{-Wall}, který způsobí,
že všechna varování pěkladač považuje za chybu a nelze je tedy ignorovat.
}.

Rozpaky, obzvláště zpočátku, budila kvalita a dostupnost nástrojů. Ta se ale v průběhu implementace zlepšila, nahrazení \textit{cabaly} modernějším
\textit{stackem} bylo podstatnou úlevou a, obávám se, na některé \uv{neobvyklosti} jsem si zvykl.

\subsection{Kvalita kódu}

Velkou výhodou se stalo i \textit{property based testování}. Je poměrně obvyklé i doporučované \citep{williams2009effectiveness} psát ke kódu i kód, který jej
automaticky testuje. To, například podle \citet{beck2003test}, snižuje množství chyb i zlepšuje kvalitu návrhu\footnote{
Přesněji řečeno, Kent Beck to tvrdí o testech, které jsou psány dříve než testovaný kód na základě poměrně jednoduché, ale striktní sady pravidel.
}. Co je podstatné, je i to,
že kvalitní a dostatečně rozsáhlé automatické může podpořit i důvěru ve výsledky.

\textit{Property based testování} přináší tuto myšlenku ještě dále. Zatímco obvykle je otestováno, že kód pro pevně dané vstupy vrátí pevně
očekávané hodnoty, je možné i otestovat, že kód pro vstupy daných vlastností vrátí výsledky daných vlastností\footnote{
Například pokud mám dva fenotypy, první vzdálenější od aktuálního optima než druhý, tak první znamená nižší fitness. To platí pro všechny takové
dvojice fenotypů.
}. Protože konkrétní vstupy nejsou součástí kódu testů, může jich testovací framework vygenerovat mnoho (ať už čistě náhodných nebo \uv{záludných})
a kód je tak otestován na mnoha různých vstupech.

Výhodou Haskellu a jeho knihovny QuickCheck \citep{Claessen:2000:QLT:351240.351266} je, že díky typovému systému velmi často není třeba předepisovat, jak má generování vstupů vypadat, a tuto práci obstará počítač. To hodnotím, ve srovnání s jinými obdobnými systémy, co jsem používal, jako značnou výhodu.

\subsection{Výkon}

Jako problematický se však ukázal výkon výsledného programu, zejména pak jeho značná paměťová náročnost.

Program je přeložen do nativního kódu, což odstraňuje část potenciálních výkonnostních probémů. Bohužel však jedna ze silných stránek
Haskellu, líné vyhodnocování, které m.j. umožňuje i snadnou tvorbu potenciálně nekonečných datových struktur, nese i negativum.
Tím je náročnost úvah o výkonu výsledného programu i tvorba nástrojů pro jeho zkoumání \citep{wadler1998no}.

To se projevilo i u \textit{fbeagle}. Ani základní analýza pomocí profileru nenašla hlavní příčinu množství alokované paměti.

Pokud se podaří tuto svízel odstranit nebo obejít, tak se tím otevře možnost spouštět větší množství simulací s větším množstvím kombinací
různých parametrů bez nutnosti spoléhat na cloud, a tedy i levněji. Dále to umožní prověřit současné výsledky spuštěním více simulací, což by
mohlo poskytnout výsledky signifikantní i pro nižší hladinu \textit{alfa.}

\subsection{Reprodukovatelnost}

Výsledný text práce je sázen prostřednictvím typografického systému \LaTeX. Můj původní plán spočíval v dotažení automatizace a
reprodukovatelnosti do extrému -- jedním příkazem by byl přeložen \textit{fbeagle}, následně spuštěny simulace, výsledky statisticky
zpracovány a vloženy do výsledného textu, z toho vygenerováno \textit{PDF}.

K tomu nedošlo, protože by kvůli nízkému výkonu \textit{fbeagle} bylo nepraktické čekat.

\section{Další možnosti}

Program \textit{fbeagle} má poměrně široké možnosti, které dosud nebyly využity k vygenerování dat vhodných pro statistickou analýzu.

Jistě má v budoucnu význam srovnat vliv pleiotropie s vlivem epistatických interakcí.

Další potenciálně zajímavá oblast parametrů jsou plynulé periodické změny prostředí a plynulý posun optimálního fenotypu jedním směrem.

\subsection{Měření evolvability}

Závažnější zásah do \textit{fbeagle}, na který bohužel nezbyl čas, je přidání měření toho, jak je populace dále schopna vývoje.
Jako vhodnější měřítko než použité směrnice aktuálního růstu na začátku zkoumaného období, se nabízí možnost simulovanou populaci
nechat simulovaně (mimo hlavní simulaci, bez vlivu na její běh, pouze jako měření vlastnosti této populace) adaptovat na
několik různých optim a u toho změřit směrnici růstu fitness.

Výhoda takovho přístupu je v tom, že neměří rychlost adaptace v jedné situaci, ale je možné změřit aktulní schopnost populace přizpůsobit
se různým podmínkám.

Dalším argumentem pro takové měření je skutečnost, že nové optimální fenotypy mohou být umístěny ve zvolené vzdálenosti od okamžitého
průměrného fenotypu populace. To je pro měření schopnosti populace se vyvýjet výhodnější, než nový optimální fenotyp umístit do 
zvolené vzdálenosti od předchozího optimálního fenotypu.
