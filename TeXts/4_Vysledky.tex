\chapter{Výsledky}

%
%V DP uveďte všechny zjištěné výsledky včetně výsledků pilotních experimentů.
% Důkladně však zvažte, jaká část vašich výsledků má být prezentována v kapitole Výsledky.
% Měly by tu být především výsledky, které čímsi přispívají k zodpovězení vašich otázek.
% Základní data (např. tabulky naměřených rozměrů nebo genotypy jedinců)  patří spíše do Příloh nebo na přiložené CD.
%
%Mějte na paměti, že obrázky spíše vyjadřují myšlenky, zatímco tabulky zobrazují data.
%Údaje z tabulek neuvádějte znovu v textu. Pozor také na duplikaci údajů obsažených v tabulkách a grafech.
%Každá tabulka a graf však musí být v textu zmíněny (zjednodušeně řečeno: tabulka ukáže data,
%která jsou dále okomentována v textu).
%
%Příklad: V tabulce číslo 1 jsou naměřené délky žížal v cm (10  10,1  10,7  10,9  11).
%V textu se pak například objeví: ...z Tabulky 1 vyplývá, že žížaly měřily od 10 do 11cm...
%
%Tabulky i obrázky by měly být očíslovány a podle čísel taky seřazeny (tedy Tabulka 1 se v textu objeví
%dříve než Tabulka 2).
%
%Věnujte také zvýšenou pozornost popiskům obrázků a tabulek, které musí být \uv{self-explanatoryi}  a zkontrolujte popis os v grafech. Tabulky by měly být co nejjednodušší. Vertikální čáry vnich nejlépe nepoužívejte vůbec a počet čar horizontálních omezte na minimum.
%
%Musí být zřejmé, nejen které statistické testy byly použity, ale také zda jsou pro ně splněny předpoklady (např. normální rozložení, pokud to test vyžaduje).
%
%Samozřejmě můžete použít barevné grafy či tabulky. Je ale dobré si uvědomit, že se hodnotí obsah, nikoliv barevnost. Klidně tedy vystačíte i jen s černou a bílou barvou (to samozřejmě nemusí platit pro fotografie nebo obrázky). Pokud použijete barevné grafy, měly by být (pokud možno) rozlišitelné i v černobílém provedení.


\section{Popisná statistika}

Nejprve uvedu hodnoty popisných statistik. Pro každou zkoumanou hodnotu podílu alel s pleiotropickým vlivem na fenotyp a
každou zkoumanou hodnotu podílu alel, které mají opačný dominantní vliv na fenotyp, než pokud se vyskytují v lokusu jen
jednou, bylo spuštěno 512 simulací. Detaily parametrů jsou popsány v příloze \ref{sec:parameters}.

V tabulkách \ref{table:vyhynuti1} až \ref{table:vyhynuti3} jsou zaznamenány počty simulací, které skončily vyhynutím
v každém úseku a nejpozději v každém úseku simulace.

\begin{table}[H]
\catcode`\-=12
\centering
    \begin{tabularx}{0.9\textwidth}{|c|Y|Y|Y|}

\cline{2-4}
\multicolumn{1}{c|}{}
 & \multicolumn{3}{c|}{Poměr pleiotropických alel} \\
\hline
 Poměr alel s neg. dominancí & 0.00 & 0.25 & 0.50 \\
\hline
 0.00                        &  89  &   86 &  92 \\
 0.05                        &  91  &   87 &  93 \\
 0.10                        &  90  &   88 &  89 \\
 0.25                        &  91  &   89 &  96 \\
\hline
\end{tabularx}
    \caption{Počet simulací končících vyhynutím v prvním úseku (t.j. v prvních 8192 krocích)}
\label{table:vyhynuti1}
\end{table}



\begin{table}[H]
\catcode`\-=12
\centering
    \begin{tabularx}{0.9\textwidth}{|c|Y|Y|Y|}

\cline{2-4}
\multicolumn{1}{c|}{}
 & \multicolumn{3}{c|}{Poměr pleiotropických alel} \\
\hline
 Poměr alel s neg. dominancí & 0.00 & 0.25 & 0.50 \\
\hline
 0.00                        & 210 (121) &  215 (129) & 220 (128) \\
 0.05                        & 218 (127) &  213 (126) & 223 (130) \\
 0.10                        & 217 (127) &  220 (132) & 219 (130) \\
 0.25                        & 222 (131) &  213 (124) & 225 (129) \\
\hline
\end{tabularx}
    \caption{Počet simulací končících vyhynutím nejpozději v druhém úseku (t.j. v prvních 16384 krocích), v závorce
    počet simulací končících vyhynutím v druhém úseku}
\label{table:vyhynuti2}
\end{table}


\begin{table}[H]
\catcode`\-=12
\centering
    \begin{tabularx}{0.9\textwidth}{|c|Y|Y|Y|}

\cline{2-4}
\multicolumn{1}{c|}{}
 & \multicolumn{3}{c|}{Poměr pleiotropických alel} \\
\hline
 Poměr alel s neg. dominancí & 0.00 & 0.25 & 0.50 \\
\hline
 0.00                        &  213 (3) &  216 (1) & 222 (2) \\
 0.05                        &  223 (5) &  215 (2) & 225 (2) \\
 0.10                        &  220 (3) &  226 (6) & 222 (3) \\
 0.25                        &  225 (3) &  216 (3) & 227 (2)\\
\hline
\end{tabularx}
\caption{Počet simulací končících vyhynutím, v závorce  počet simulací končících vyhynutím v třetím úseku}
\label{table:vyhynuti3}
\end{table}

%%%%%%%%%%%%%%%%%%%%%%%%%%%%%%%%%%%%%%%%%%%%


Základní popisné statistiky pro dosažená maxima fitness v jednotlivých úsecích simulace
jsou zapsány v tabulkách \ref{table:max1} až \ref{table:max3}. Pro každou je uveden aritmetický průměr, medián a
směrodatná odchylka.

\begin{table}[H]
\catcode`\-=12
\centering
    \begin{tabularx}{1.0\textwidth}{|c|Y|Y|Y|Y|Y|Y|Y|Y|Y|}

\cline{2-10}
\multicolumn{1}{c|}{}
 & \multicolumn{9}{c|}{Poměr pleiotropických alel} \\
\hline
\multirowcell{2}{Podíl neg. \\ dom. alel} & \multicolumn{3}{c|}{0.00} & \multicolumn{3}{c|}{0.25} & \multicolumn{3}{c|}{0.50} \\
\cline{2-10}
        & avg & med & s.d. & avg & med & s.d. & avg & med & s.d. \\
\hline
 0.00                        & 0.792 & 0.961 & 0.357 & 0.800 & 0.962 & 0.353 & 0.789 & 0.961 & 0.362 \\
 0.05                        & 0.789 & 0.960 & 0.360 & 0.796 & 0.961 & 0.359 & 0.787 & 0.961 & 0.363 \\
 0.10                        & 0.790 & 0.960 & 0.358 & 0.796 & 0.962 & 0.356 & 0.786 & 0.958 & 0.359 \\
 0.25                        & 0.786 & 0.958 & 0.359 & 0.791 & 0.958 & 0.356 & 0.780 & 0.959 & 0.366 \\
\hline
\end{tabularx}
\caption{Nejvyšší dosažená  průměrná fitness v prvním úseku (t.j. v prvních 8192 krocích)}
\label{table:max1}
\end{table}

\begin{table}[H]
\catcode`\-=12
\centering
    \begin{tabularx}{1.0\textwidth}{|c|Y|Y|Y|Y|Y|Y|Y|Y|Y|}

\cline{2-10}
\multicolumn{1}{c|}{}
 & \multicolumn{9}{c|}{Poměr pleiotropických alel} \\
\hline
\multirowcell{2}{Podíl neg. \\ dom. alel} & \multicolumn{3}{c|}{0.00} & \multicolumn{3}{c|}{0.25} & \multicolumn{3}{c|}{0.50} \\
\cline{2-10}
        & avg & med & s.d. & avg & med & s.d. & avg & med & s.d. \\
\hline
 0.00                        & 0.631 & 0.864 & 0.404 & 0.615 & 0.852 & 0.410 & 0.616 & 0.858 & 0.410 \\
 0.05                        & 0.612 & 0.848 & 0.408 & 0.620 & 0.849 & 0.407 & 0.612 & 0.859 & 0.413 \\
 0.10                        & 0.611 & 0.841 & 0.407 & 0.607 & 0.846 & 0.412 & 0.598 & 0.820 & 0.409 \\
 0.25                        & 0.598 & 0.820 & 0.409 & 0.613 & 0.830 & 0.402 & 0.603 & 0.843 & 0.410 \\
\hline
\end{tabularx}
\caption{Nejvyšší dosažená  průměrná fitness v druhém úseku (t.j. v krocích 8193--16384)}
\label{table:max2}
\end{table}

\begin{table}[H]
\catcode`\-=12
\centering
    \begin{tabularx}{1.0\textwidth}{|c|Y|Y|Y|Y|Y|Y|Y|Y|Y|}

\cline{2-10}
\multicolumn{1}{c|}{}
 & \multicolumn{9}{c|}{Poměr pleiotropických alel} \\
\hline
\multirowcell{2}{Podíl neg. \\ dom. alel} & \multicolumn{3}{c|}{0.00} & \multicolumn{3}{c|}{0.25} & \multicolumn{3}{c|}{0.50} \\
\cline{2-10}
        & avg & med & s.d. & avg & med & s.d. & avg & med & s.d. \\
\hline
 0.00                        & 0.956 & 0.972 & 0.094 & 0.962 & 0.972 & 0.058 & 0.959 & 0.971 & 0.080 \\
 0.05                        & 0.948 & 0.971 & 0.120 & 0.958 & 0.971 & 0.079 & 0.958 & 0.979 & 0.081 \\
 0.10                        & 0.954 & 0.971 & 0.133 & 0.945 & 0.970 & 0.133 & 0.952 & 0.970 & 0.097 \\
 0.25                        & 0.952 & 0.970 & 0.097 & 0.953 & 0.969 & 0.095 & 0.957 & 0.970 & 0.080 \\
\hline
\end{tabularx}
\caption{Nejvyšší dosažená  průměrná fitness v třetím úseku (t.j. v kroku 16385 a následujících)}
\label{table:max3}
\end{table}

%%%%%%%%%%%%%%%%%%%%%%%%%%

Základní popisné statistiky pro směrnice růstu průměrné fitness na počátcích a koncích jednotlivých úseků simulace
jsou zaznamenány v tabulkách \ref{table:sm1} až \ref{table:sem3}. Pro každou je uveden aritmetický průměr, medián a
směrodatná odchylka.

\begin{table}[H]
\catcode`\-=12
\centering
    \begin{tabularx}{1.0\textwidth}{|c|Y|Y|Y|Y|Y|Y|Y|Y|Y|}

\cline{2-10}
\multicolumn{1}{c|}{}
 & \multicolumn{9}{c|}{Poměr pleiotropických alel} \\
\hline
\multirowcell{2}{Podíl neg. \\ dom. alel} & \multicolumn{3}{c|}{0.00} & \multicolumn{3}{c|}{0.25} & \multicolumn{3}{c|}{0.50} \\
\cline{2-10}
                              & avg   & med   & s.d.  & avg   & med   & s.d.  & avg   & med   & s.d. \\
\hline
  0.00                        & 2.770 & 2.586 & 1.253 & 2.804 & 2.740 & 1.022 & 2.773 & 2.715 & 1.050 \\
  0.05                        & 2.684 & 2.575 & 1.015 & 2.717 & 2.654 & 0.963 & 2.717 & 2.632 & 1.023 \\
  0.10                        & 2.633 & 2.549 & 0.962 & 2.655 & 2.583 & 0.920 & 2.660 & 2.591 & 0.944 \\
  0.25                        & 2.550 & 2.444 & 0.905 & 2.611 & 2.500 & 0.948 & 2.578 & 2.502 & 0.911 \\
\hline
\end{tabularx}
\caption{Směrnice růstu průměrné fitness na počátku prvního úseku}
\label{table:sm1}
\end{table}


\begin{table}[H]
\catcode`\-=12
\centering
    \begin{tabularx}{1.0\textwidth}{|c|Y|Y|Y|Y|Y|Y|Y|Y|Y|}

\cline{2-10}
\multicolumn{1}{c|}{}
 & \multicolumn{9}{c|}{Poměr pleiotropických alel} \\
\hline
\multirowcell{2}{Podíl neg. \\ dom. alel} & \multicolumn{3}{c|}{0.00} & \multicolumn{3}{c|}{0.25} & \multicolumn{3}{c|}{0.50} \\
\cline{2-10}
                              & avg   & med   & s.d.  & avg   & med   & s.d.  & avg   & med   & s.d. \\
\hline
  0.00                        & 1.173 & 1.143 & 0.219 & 1.161 & 1.143 & 0.180 & 1.117 & 1.138 & 0.197 \\
  0.05                        & 1.168 & 1.137 & 0.181 & 1.176 & 1.131 & 0.653 & 1.165 & 1.136 & 0.206 \\
  0.10                        & 1.164 & 1.135 & 0.193 & 1.162 & 1.271 & 0.320 & 1.161 & 1.128 & 0.237 \\
  0.25                        & 1.165 & 1.121 & 0.227 & 1.170 & 1.130 & 0.266 & 1.180 & 1.131 & 0.336 \\
\hline
\end{tabularx}
\caption{Směrnice růstu průměrné fitness na počátku druhého úseku}
\label{table:sm2}
\end{table}

\begin{table}[H]
\catcode`\-=12
\centering
    \begin{tabularx}{1.0\textwidth}{|c|Y|Y|Y|Y|Y|Y|Y|Y|Y|}

\cline{2-10}
\multicolumn{1}{c|}{}
 & \multicolumn{9}{c|}{Poměr pleiotropických alel} \\
\hline
\multirowcell{2}{Podíl neg. \\ dom. alel} & \multicolumn{3}{c|}{0.00} & \multicolumn{3}{c|}{0.25} & \multicolumn{3}{c|}{0.50} \\
\cline{2-10}
        & avg & med & s.d. & avg & med & s.d. & avg & med & s.d. \\
\hline
  0.00                        & 1.373 & 1.302 & 0.306 & 1.368 & 1.281 & 0.297 & 1.323 & 1.281 & 0.211 \\
  0.05                        & 1.363 & 1.286 & 0.321 & 1.351 & 1.290 & 0.286 & 1.332 & 1.279 & 0.288 \\
  0.10                        & 1.358 & 1.287 & 0.272 & 1.347 & 1.286 & 0.288 & 1.327 & 1.267 & 0.302 \\
  0.25                        & 1.359 & 1.263 & 0.385 & 1.356 & 1.307 & 0.306 & 1.355 & 1.265 & 0.541 \\
\hline
\end{tabularx}
\caption{Směrnice růstu průměrné fitness na počátku třetího úseku}
\label{table:sm3}
\end{table}


%%%%%%%%%%%%%%%%%%%%%%%%%%5

\begin{table}[H]
\catcode`\-=12
\centering
    \begin{tabularx}{1.0\textwidth}{|c|Y|Y|Y|Y|Y|Y|Y|Y|Y|}

\cline{2-10}
\multicolumn{1}{c|}{}
 & \multicolumn{9}{c|}{Poměr pleiotropických alel} \\
\hline
\multirowcell{2}{Podíl neg. \\ dom. alel} & \multicolumn{3}{c|}{0.00} & \multicolumn{3}{c|}{0.25} & \multicolumn{3}{c|}{0.50} \\
\cline{2-10}
        & avg & med & s.d. & avg & med & s.d. & avg & med & s.d. \\
\hline
  0.00                        & 0. & 0. & 0. & 0. & 0. & 0. & 0. & 0. & 0. \\
  0.05                        & 0. & 0. & 0. & 0. & 0. & 0. & 0. & 0. & 0. \\
  0.10                        & 0. & 0. & 0. & 0. & 0. & 0. & 0. & 0. & 0. \\
  0.25                        & 0. & 0. & 0. & 0. & 0. & 0. & 0. & 0. & 0. \\
\hline
\end{tabularx}
\caption{Směrnice růstu průměrné fitness na konci prvního úseku}
\label{table:sem1}
\end{table}



\begin{table}[H]
\catcode`\-=12
\centering
    \begin{tabularx}{1.0\textwidth}{|c|Y|Y|Y|Y|Y|Y|Y|Y|Y|}

\cline{2-10}
\multicolumn{1}{c|}{}
 & \multicolumn{9}{c|}{Poměr pleiotropických alel} \\
\hline
\multirowcell{2}{Podíl neg. \\ dom. alel} & \multicolumn{3}{c|}{0.00} & \multicolumn{3}{c|}{0.25} & \multicolumn{3}{c|}{0.50} \\
\cline{2-10}
        & avg & med & s.d. & avg & med & s.d. & avg & med & s.d. \\
\hline
  0.00                        & 0. & 0. & 0. & 0. & 0. & 0. & 0. & 0. & 0. \\
  0.05                        & 0. & 0. & 0. & 0. & 0. & 0. & 0. & 0. & 0. \\
  0.10                        & 0. & 0. & 0. & 0. & 0. & 0. & 0. & 0. & 0. \\
  0.25                        & 0. & 0. & 0. & 0. & 0. & 0. & 0. & 0. & 0. \\
\hline
\end{tabularx}
\caption{Směrnice růstu průměrné fitness na konci druhého úseku}
\label{table:sem2}
\end{table}

\begin{table}[H]
\catcode`\-=12
\centering
    \begin{tabularx}{1.0\textwidth}{|c|Y|Y|Y|Y|Y|Y|Y|Y|Y|}

\cline{2-10}
\multicolumn{1}{c|}{}
 & \multicolumn{9}{c|}{Poměr pleiotropických alel} \\
\hline
\multirowcell{2}{Podíl neg. \\ dom. alel} & \multicolumn{3}{c|}{0.00} & \multicolumn{3}{c|}{0.25} & \multicolumn{3}{c|}{0.50} \\
\cline{2-10}
        & avg & med & s.d. & avg & med & s.d. & avg & med & s.d. \\
\hline
  0.00                        & 0. & 0. & 0. & 0. & 0. & 0. & 0. & 0. & 0. \\
  0.05                        & 0. & 0. & 0. & 0. & 0. & 0. & 0. & 0. & 0. \\
  0.10                        & 0. & 0. & 0. & 0. & 0. & 0. & 0. & 0. & 0. \\
  0.25                        & 0. & 0. & 0. & 0. & 0. & 0. & 0. & 0. & 0. \\
\hline
\end{tabularx}
\caption{Směrnice růstu průměrné fitness na konci třetího úseku}
\label{table:sem3}
\end{table}


%%%%%%%%%%%%%%%%%%%%%%

\begin{table}[H]
\catcode`\-=12
\centering
    \begin{tabularx}{1.0\textwidth}{|c|Y|Y|Y|Y|Y|Y|Y|Y|Y|}

\cline{2-10}
\multicolumn{1}{c|}{}
 & \multicolumn{9}{c|}{Poměr pleiotropických alel} \\
\hline
\multirowcell{2}{Podíl neg. \\ dom. alel} & \multicolumn{3}{c|}{0.00} & \multicolumn{3}{c|}{0.25} & \multicolumn{3}{c|}{0.50} \\
\cline{2-10}
        & avg & med & s.d. & avg & med & s.d. & avg & med & s.d. \\
\hline
 0.00                        & 0. & 0. & 0. & 0. & 0. & 0. & 0. & 0. & 0. \\
 0.05                        & 0. & 0. & 0. & 0. & 0. & 0. & 0. & 0. & 0. \\
 0.10                        & 0. & 0. & 0. & 0. & 0. & 0. & 0. & 0. & 0. \\
 0.25                        & 0. & 0. & 0. & 0. & 0. & 0. & 0. & 0. & 0. \\
\hline
\end{tabularx}
\caption{Desetiprocentní percentil průměrné fitness v prvním úseku (t.j. v prvních 8192 krocích)}
\end{table}



\begin{table}[H]
\catcode`\-=12
\centering
    \begin{tabularx}{1.0\textwidth}{|c|Y|Y|Y|Y|Y|Y|Y|Y|Y|}

\cline{2-10}
\multicolumn{1}{c|}{}
 & \multicolumn{9}{c|}{Poměr pleiotropických alel} \\
\hline
\multirowcell{2}{Podíl neg. \\ dom. alel} & \multicolumn{3}{c|}{0.00} & \multicolumn{3}{c|}{0.25} & \multicolumn{3}{c|}{0.50} \\
\cline{2-10}
        & avg & med & s.d. & avg & med & s.d. & avg & med & s.d. \\
\hline
 0.00                       & 0. & 0. & 0. & 0. & 0. & 0. & 0. & 0. & 0. \\
 0.05                       & 0. & 0. & 0. & 0. & 0. & 0. & 0. & 0. & 0. \\
 0.10                       & 0. & 0. & 0. & 0. & 0. & 0. & 0. & 0. & 0. \\
 0.25                       & 0. & 0. & 0. & 0. & 0. & 0. & 0. & 0. & 0. \\
\hline
\end{tabularx}
\caption{Desetiprocentní percentil průměrné fitness v druhém úseku (t.j. v krocích 8193--16384)}
\end{table}

\begin{table}[H]
\catcode`\-=12
\centering
    \begin{tabularx}{1.0\textwidth}{|c|Y|Y|Y|Y|Y|Y|Y|Y|Y|}

\cline{2-10}
\multicolumn{1}{c|}{}
 & \multicolumn{9}{c|}{Poměr pleiotropických alel} \\
\hline
\multirowcell{2}{Podíl neg. \\ dom. alel} & \multicolumn{3}{c|}{0.00} & \multicolumn{3}{c|}{0.25} & \multicolumn{3}{c|}{0.50} \\
\cline{2-10}
        & avg & med & s.d. & avg & med & s.d. & avg & med & s.d. \\
\hline
 0.00                        & 0. & 0. & 0. & 0. & 0. & 0. & 0. & 0. & 0. \\
 0.05                        & 0. & 0. & 0. & 0. & 0. & 0. & 0. & 0. & 0. \\
 0.10                        & 0. & 0. & 0. & 0. & 0. & 0. & 0. & 0. & 0. \\
 0.25                        & 0. & 0. & 0. & 0. & 0. & 0. & 0. & 0. & 0. \\
\hline
\end{tabularx}
\caption{Desetiprocentní percentil průměrné fitness v třetím úseku (t.j. v kroku 16385 a následujících)}
\end{table}

\section{Název druhé podkapitoly v druhé kapitole}

lore ipsum XXXX