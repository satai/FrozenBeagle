\chapter{Výsledky}

%
%V DP uveďte všechny zjištěné výsledky včetně výsledků pilotních experimentů.
% Důkladně však zvažte, jaká část vašich výsledků má být prezentována v kapitole Výsledky.
% Měly by tu být především výsledky, které čímsi přispívají k zodpovězení vašich otázek.
% Základní data (např. tabulky naměřených rozměrů nebo genotypy jedinců)  patří spíše do Příloh nebo na přiložené CD.
%
%Mějte na paměti, že obrázky spíše vyjadřují myšlenky, zatímco tabulky zobrazují data.
%Údaje z tabulek neuvádějte znovu v textu. Pozor také na duplikaci údajů obsažených v tabulkách a grafech.
%Každá tabulka a graf však musí být v textu zmíněny (zjednodušeně řečeno: tabulka ukáže data,
%která jsou dále okomentována v textu).
%
%Příklad: V tabulce číslo 1 jsou naměřené délky žížal v cm (10  10,1  10,7  10,9  11).
%V textu se pak například objeví: ...z Tabulky 1 vyplývá, že žížaly měřily od 10 do 11cm...
%
%Tabulky i obrázky by měly být očíslovány a podle čísel taky seřazeny (tedy Tabulka 1 se v textu objeví
%dříve než Tabulka 2).
%
%Věnujte také zvýšenou pozornost popiskům obrázků a tabulek, které musí být \uv{self-explanatoryi}  a zkontrolujte popis os v grafech. Tabulky by měly být co nejjednodušší. Vertikální čáry vnich nejlépe nepoužívejte vůbec a počet čar horizontálních omezte na minimum.
%
%Musí být zřejmé, nejen které statistické testy byly použity, ale také zda jsou pro ně splněny předpoklady (např. normální rozložení, pokud to test vyžaduje).
%
%Samozřejmě můžete použít barevné grafy či tabulky. Je ale dobré si uvědomit, že se hodnotí obsah, nikoliv barevnost. Klidně tedy vystačíte i jen s černou a bílou barvou (to samozřejmě nemusí platit pro fotografie nebo obrázky). Pokud použijete barevné grafy, měly by být (pokud možno) rozlišitelné i v černobílém provedení.

Pro lepší představu o fungování modelu jsou v příloze XXX reprezentativní grafy pro průběhy některých veličin,
včetně těch, které nebyly součástí hypotéz, ale mohou pomoci s porozuměním.

\section{Vymírání populací}

Pro každou zkoumanou hodnotu podílu alel s pleiotropickým vlivem na
fenotyp a každou zkoumanou hodnotu podílu alel, které mají opačný dominantní vliv na fenotyp, než pokud se
vyskytují v lokusu jen jednou, bylo spuštěno 512 simulací. Detaily parametrů jsou popsány v příloze
\ref{sec:parameters}.

\subsection{Popisná statistika}

V tabulkách \ref{table:vyhynuti1} až \ref{table:vyhynuti3} jsou zaznamenány počty simulací, které skončily vyhynutím
v daném úseku a nejpozději v daném úseku simulace.

\begin{table}[H]
\catcode`\-=12
\centering
    \begin{tabularx}{0.9\textwidth}{|c|Y|Y|Y|}

\cline{2-4}
\multicolumn{1}{c|}{}
 & \multicolumn{3}{c|}{Poměr pleiotropických alel} \\
\hline
 Poměr alel s neg. dominancí & 0.00 & 0.25 & 0.50 \\
\hline
 0.00                        &  89  &   86 &  92 \\
 0.05                        &  91  &   87 &  93 \\
 0.10                        &  90  &   88 &  89 \\
 0.25                        &  91  &   89 &  96 \\
\hline
\end{tabularx}
    \caption{Počet simulací končících vyhynutím v prvním úseku (t.j. v prvních 8192 krocích)}
\label{table:vyhynuti1}
\end{table}

Je patrné, že po první změně optima vymírá více populací, než na počátku simulace:

\begin{table}[H]
\catcode`\-=12
\centering
    \begin{tabularx}{0.9\textwidth}{|c|Y|Y|Y|}

\cline{2-4}
\multicolumn{1}{c|}{}
 & \multicolumn{3}{c|}{Poměr pleiotropických alel} \\
\hline
 Poměr alel s neg. dominancí & 0.00 & 0.25 & 0.50 \\
\hline
 0.00                        & 210 (121) &  215 (129) & 220 (128) \\
 0.05                        & 218 (127) &  213 (126) & 223 (130) \\
 0.10                        & 217 (127) &  220 (132) & 219 (130) \\
 0.25                        & 222 (131) &  213 (124) & 225 (129) \\
\hline
\end{tabularx}
    \caption{Počet simulací končících vyhynutím nejpozději v druhém úseku (t.j. v prvních 16384 krocích), v závorce
    počet simulací končících vyhynutím v druhém úseku}
\label{table:vyhynuti2}
\end{table}

Z populací, které přežily první změnu optimálního fenotypu, již po návratu optima vyhyne poměrně málo:

\begin{table}[H]
\catcode`\-=12
\centering
    \begin{tabularx}{0.9\textwidth}{|c|Y|Y|Y|}

\cline{2-4}
\multicolumn{1}{c|}{}
 & \multicolumn{3}{c|}{Poměr pleiotropických alel} \\
\hline
 Poměr alel s neg. dominancí & 0.00 & 0.25 & 0.50 \\
\hline
 0.00                        &  213 (3) &  216 (1) & 222 (2) \\
 0.05                        &  223 (5) &  215 (2) & 225 (2) \\
 0.10                        &  220 (3) &  226 (6) & 222 (3) \\
 0.25                        &  225 (3) &  216 (3) & 227 (2)\\
\hline
\end{tabularx}
\caption{Počet simulací končících vyhynutím, v závorce  počet simulací končících vyhynutím v třetím úseku}
\label{table:vyhynuti3}
\end{table}

\subsection{Otestování hypotéz}

Vyhynutí populace se týkaly následující hypotézy:

\begin{enumerate}
    \item{Populace, kde je nenulové zastoupení negativně dominantních alel nebo alel ovlivňujících více složek fenotypu,
          častěji vymírají.}
    \item{Efekty obou typů alel se vzájemně multiplikují.}
\end{enumerate}

XXXXXXXXXXXX

%%%%%%%%%%%%%%%%%%%%%%%%%%%%%%%%%%%%%%%%%%%%

\section{Dosažená průměrná fitness}

\subsection{Popisná statistika}

Základní popisné statistiky pro dosažená maxima průměrů fitness, tedy jak dobře se průměru populace podařilo
přizpůsobit aktuálním podmínkám, v jednotlivých úsecích simulace jsou zapsány v tabulkách \ref{table:max1}
až \ref{table:max3}. Pro každou je uveden aritmetický průměr, medián a směrodatná odchylka.

\begin{table}[H]
\catcode`\-=12
\centering
    \begin{tabularx}{1.0\textwidth}{|c|Y|Y|Y|Y|Y|Y|Y|Y|Y|}

\cline{2-10}
\multicolumn{1}{c|}{}
 & \multicolumn{9}{c|}{Poměr pleiotropických alel} \\
\hline
\multirowcell{2}{Podíl neg. \\ dom. alel} & \multicolumn{3}{c|}{0.00} & \multicolumn{3}{c|}{0.25} & \multicolumn{3}{c|}{0.50} \\
\cline{2-10}
        & avg & med & s.d. & avg & med & s.d. & avg & med & s.d. \\
\hline
 0.00                        & 0.792 & 0.961 & 0.357 & 0.800 & 0.962 & 0.353 & 0.789 & 0.961 & 0.362 \\
 0.05                        & 0.789 & 0.960 & 0.360 & 0.796 & 0.961 & 0.359 & 0.787 & 0.961 & 0.363 \\
 0.10                        & 0.790 & 0.960 & 0.358 & 0.796 & 0.962 & 0.356 & 0.786 & 0.958 & 0.359 \\
 0.25                        & 0.786 & 0.958 & 0.359 & 0.791 & 0.958 & 0.356 & 0.780 & 0.959 & 0.366 \\
\hline
\end{tabularx}
\caption{Nejvyšší dosažená  průměrná fitness v prvním úseku (t.j. v prvních 8192 krocích)}
\label{table:max1}
\end{table}

Průměry jsou po změně optima viditelně menší:

\begin{table}[H]
\catcode`\-=12
\centering
    \begin{tabularx}{1.0\textwidth}{|c|Y|Y|Y|Y|Y|Y|Y|Y|Y|}

\cline{2-10}
\multicolumn{1}{c|}{}
 & \multicolumn{9}{c|}{Poměr pleiotropických alel} \\
\hline
\multirowcell{2}{Podíl neg. \\ dom. alel} & \multicolumn{3}{c|}{0.00} & \multicolumn{3}{c|}{0.25} & \multicolumn{3}{c|}{0.50} \\
\cline{2-10}
        & avg & med & s.d. & avg & med & s.d. & avg & med & s.d. \\
\hline
 0.00                        & 0.631 & 0.864 & 0.404 & 0.615 & 0.852 & 0.410 & 0.616 & 0.858 & 0.410 \\
 0.05                        & 0.612 & 0.848 & 0.408 & 0.620 & 0.849 & 0.407 & 0.612 & 0.859 & 0.413 \\
 0.10                        & 0.611 & 0.841 & 0.407 & 0.607 & 0.846 & 0.412 & 0.598 & 0.820 & 0.409 \\
 0.25                        & 0.598 & 0.820 & 0.409 & 0.613 & 0.830 & 0.402 & 0.603 & 0.843 & 0.410 \\
\hline
\end{tabularx}
\caption{Nejvyšší dosažená  průměrná fitness v druhém úseku (t.j. v krocích 8193--16384)}
\label{table:max2}
\end{table}

Ve třetím období jsou populace nejlépe přizpůsobené. Navíc je rozptyl těchto hodnot nejmenší.

\begin{table}[H]
\catcode`\-=12
\centering
    \begin{tabularx}{1.0\textwidth}{|c|Y|Y|Y|Y|Y|Y|Y|Y|Y|}

\cline{2-10}
\multicolumn{1}{c|}{}
 & \multicolumn{9}{c|}{Poměr pleiotropických alel} \\
\hline
\multirowcell{2}{Podíl neg. \\ dom. alel} & \multicolumn{3}{c|}{0.00} & \multicolumn{3}{c|}{0.25} & \multicolumn{3}{c|}{0.50} \\
\cline{2-10}
        & avg & med & s.d. & avg & med & s.d. & avg & med & s.d. \\
\hline
 0.00                        & 0.956 & 0.972 & 0.094 & 0.962 & 0.972 & 0.058 & 0.959 & 0.971 & 0.080 \\
 0.05                        & 0.948 & 0.971 & 0.120 & 0.958 & 0.971 & 0.079 & 0.958 & 0.979 & 0.081 \\
 0.10                        & 0.954 & 0.971 & 0.133 & 0.945 & 0.970 & 0.133 & 0.952 & 0.970 & 0.097 \\
 0.25                        & 0.952 & 0.970 & 0.097 & 0.953 & 0.969 & 0.095 & 0.957 & 0.970 & 0.080 \\
\hline
\end{tabularx}
\caption{Nejvyšší dosažená průměrná fitness v třetím úseku (t.j. v kroku 16385 a následujících)}
\label{table:max3}
\end{table}

\subsection{Otestování hypotéz}

Průměry fitness figurují v prvních částech následujících hypotéz:

\begin{enumerate}
    \setcounter{enumi}{2}
    \item{Populace, kde je nenulové zastoupení negativně dominantních alel a zároveň dostatečné množství
          alel ovlivňujících více složek fenotypu, dosahují menších průměrů i percentilu fitness (t.j.
          nedostanou se tak blízko optimu).}
    \item{Tento efekt negativně dominantních alel roste s množstvím alel ovlivňujících více složek fenotypu.}
\end{enumerate}

Část třetí hypotézy týkající se průměrů fitness výsledky lineárního modelu (p hodnota pro vliv dominance je 0.744 pro první úsek,
0.346 pro druhý úsek a 0.571 pro třetí; p hodnota pro vliv pleiotropie 0.956 pro první, 0.648 pro druhý a
0.442 pro třetí úsek; p hodnoty pro interakci dominance a pleiotropie jsou 0.920, 0.676 a 0.976) nepotvrdily.
To automaticky znamená i zamítnutí té části čtvrté hypotézy, která se týká průměrů fitness.

%%%%%%%%%%%%%%%%%%%%%%

\section{Desetiprocentní percentily fitness}

Základní popisné statistiky pro desetiprocentní percentil fitness v jednotlivých úsecích simulace
jsou zaznamenány v tabulkách \ref{table:per1} až \ref{table:per3}. Pro každou je uveden aritmetický průměr, medián a
směrodatná odchylka.

Desetiprocentní percentil fitness XXX

\begin{table}[H]
\catcode`\-=12
\centering
    \begin{tabularx}{1.0\textwidth}{|c|Y|Y|Y|Y|Y|Y|Y|Y|Y|}

\cline{2-10}
\multicolumn{1}{c|}{}
 & \multicolumn{9}{c|}{Poměr pleiotropických alel} \\
\hline
\multirowcell{2}{Podíl neg. \\ dom. alel} & \multicolumn{3}{c|}{0.00} & \multicolumn{3}{c|}{0.25} & \multicolumn{3}{c|}{0.50} \\
\cline{2-10}
        & avg & med & s.d. & avg & med & s.d. & avg & med & s.d. \\
\hline
 0.00                        & 0.764 & 0.928 & 0.348 & 0.772 & 0.931 & 0.344 & 0.760 & 0.926 & 0.352 \\
 0.05                        & 0.762 & 0.928 & 0.351 & 0.768 & 0.929 & 0.345 & 0.759 & 0.928 & 0.354 \\
 0.10                        & 0.762 & 0.929 & 0.350 & 0.769 & 0.932 & 0.347 & 0.764 & 0.927 & 0.348 \\
 0.25                        & 0.764 & 0.926 & 0.348 & 0.762 & 0.926 & 0.327 & 0.750 & 0.926 & 0.357 \\
\hline
\end{tabularx}
\caption{Desetiprocentní percentil průměrné fitness v prvním úseku (t.j. v prvních 8192 krocích)}
\label{table:per1}
\end{table}

Podobně jako u průměrů i desetiprocentní percentily jsou nejmenší po první změně optima:

\begin{table}[H]
\catcode`\-=12
\centering
    \begin{tabularx}{1.0\textwidth}{|c|Y|Y|Y|Y|Y|Y|Y|Y|Y|}

\cline{2-10}
\multicolumn{1}{c|}{}
 & \multicolumn{9}{c|}{Poměr pleiotropických alel} \\
\hline
\multirowcell{2}{Podíl neg. \\ dom. alel} & \multicolumn{3}{c|}{0.00} & \multicolumn{3}{c|}{0.25} & \multicolumn{3}{c|}{0.50} \\
\cline{2-10}
        & avg & med & s.d. & avg & med & s.d. & avg & med & s.d. \\
\hline
 0.00                       & 0.582 & 0.772 & 0.386 & 0.564 & 0.750 & 0.390 & 0.567 & 0.762 & 0.390 \\
 0.05                       & 0.564 & 0.752 & 0.389 & 0.570 & 0.753 & 0.387 & 0.564 & 0.768 & 0.393 \\
 0.10                       & 0.562 & 0.743 & 0.388 & 0.558 & 0.746 & 0.391 & 0.559 & 0.756 & 0.390 \\
 0.25                       & 0.549 & 0.718 & 0.390 & 0.561 & 0.727 & 0.383 & 0.554 & 0.749 & 0.391 \\
\hline
\end{tabularx}
\caption{Desetiprocentní percentil průměrné fitness v druhém úseku (t.j. v krocích 8193--16384)}
\label{table:per2}
\end{table}

Tak jako průměry, tak i desetiprocentní percentily dosáhnou nejvyšší hodnoty ve třetím období. V tomto období jsou také
patrné jejich nejmenší rozptyly:

\begin{table}[H]
\catcode`\-=12
\centering
    \begin{tabularx}{1.0\textwidth}{|c|Y|Y|Y|Y|Y|Y|Y|Y|Y|}

\cline{2-10}
\multicolumn{1}{c|}{}
 & \multicolumn{9}{c|}{Poměr pleiotropických alel} \\
\hline
\multirowcell{2}{Podíl neg. \\ dom. alel} & \multicolumn{3}{c|}{0.00} & \multicolumn{3}{c|}{0.25} & \multicolumn{3}{c|}{0.50} \\
\cline{2-10}
        & avg & med & s.d. & avg & med & s.d. & avg & med & s.d. \\
\hline
 0.00                        & 0.928 & 0.948 & 0.096 & 0.934 & 0.947 & 0.064 & 0.930 & 0.947 & 0.083 \\
 0.05                        & 0.921 & 0.947 & 0.120 & 0.930 & 0.946 & 0.082 & 0.929 & 0.945 & 0.084 \\
 0.10                        & 0.927 & 0.948 & 0.097 & 0.924 & 0.944 & 0.097 & 0.924 & 0.944 & 0.100 \\
 0.25                        & 0.923 & 0.944 & 0.100 & 0.923 & 0.944 & 0.098 & 0.928 & 0.947 & 0.083 \\
\hline
\end{tabularx}
\caption{Desetiprocentní percentil průměrné fitness v třetím úseku (t.j. v kroku 16385 a následujících)}
\label{table:per3}
\end{table}

%%%%%%%%%%%%%%%%%%%%%%%%%%

\section{Růst fitness}

\subsection{Počátky období}

Základní popisné statistiky pro směrnice růstu průměrné fitness na počátcích jednotlivých úseků simulace
jsou zaznamenány v tabulkách \ref{table:sm1} až \ref{table:sm3}. Pro každou je uveden aritmetický průměr, medián a
směrodatná odchylka.

\begin{table}[H]
\scriptsize
\catcode`\-=12
\centering
    \begin{tabularx}{1.0\textwidth}{|c|Y|Y|Y|Y|Y|Y|Y|Y|Y|}

\cline{2-10}
\multicolumn{1}{c|}{}
 & \multicolumn{9}{c|}{Poměr pleiotropických alel} \\
\hline
\multirowcell{2}{Podíl neg. \\ dom. alel} & \multicolumn{3}{c|}{0.00} & \multicolumn{3}{c|}{0.25} & \multicolumn{3}{c|}{0.50} \\
\cline{2-10}
                              & avg   & med   & s.d.  & avg   & med   & s.d.  & avg   & med   & s.d. \\
\hline
  0.00                        & 0.02164 & 0.02020 & 0.00879 & 0.02190 & 0.02140 & 0.00798 & 0.02166 & 0.02121 & 0.00820 \\
  0.05                        & 0.02097 & 0.02011 & 0.00793 & 0.02123 & 0.02073 & 0.00752 & 0.02123 & 0.02056 & 0.00799 \\
  0.10                        & 0.02057 & 0.01991 & 0.00752 & 0.02074 & 0.02018 & 0.00710 & 0.02078 & 0.02025 & 0.00738 \\
  0.25                        & 0.01992 & 0.01909 & 0.00707 & 0.02040 & 0.01951 & 0.00740 & 0.02014 & 0.01955 & 0.00712 \\
\hline
\end{tabularx}
\caption{Směrnice růstu průměrné fitness na počátku prvního úseku}
\label{table:sm1}
\end{table}

Po změně optima rostou fitness v populaci dříve adaptované na jiné podmínky pomaleji:

\begin{table}[H]
\scriptsize
\catcode`\-=12
\centering
    \begin{tabularx}{1.0\textwidth}{|c|Y|Y|Y|Y|Y|Y|Y|Y|Y|}

\cline{2-10}
\multicolumn{1}{c|}{}
 & \multicolumn{9}{c|}{Poměr pleiotropických alel} \\
\hline
\multirowcell{2}{Podíl neg. \\ dom. alel} & \multicolumn{3}{c|}{0.00} & \multicolumn{3}{c|}{0.25} & \multicolumn{3}{c|}{0.50} \\
\cline{2-10}
                              & avg   & med   & s.d.  & avg   & med   & s.d.  & avg   & med   & s.d. \\
\hline
  0.00                        & 0.00917 & 0.00893 & 0.00171 & 0.00907 & 0.00893 & 0.00141 & 0.00919 & 0.00889 & 0.00154 \\
  0.05                        & 0.00913 & 0.00888 & 0.00141 & 0.00919 & 0.00883 & 0.00276 & 0.00910 & 0.00888 & 0.00161 \\
  0.10                        & 0.00910 & 0.00887 & 0.00151 & 0.00908 & 0.00881 & 0.00249 & 0.01037 & 0.00990 & 0.00185 \\
  0.25                        & 0.00910 & 0.00876 & 0.00177 & 0.00914 & 0.00883 & 0.00207 & 0.01059 & 0.00989 & 0.00262 \\
\hline
\end{tabularx}
\caption{Směrnice růstu průměrné fitness na počátku druhého úseku}
\label{table:sm2}
\end{table}

Po změně optima na původní rostou fitness také pomaleji než v prvním období:

\begin{table}[H]
\scriptsize
\catcode`\-=12
\centering
    \begin{tabularx}{1.0\textwidth}{|c|Y|Y|Y|Y|Y|Y|Y|Y|Y|}

\cline{2-10}
\multicolumn{1}{c|}{}
 & \multicolumn{9}{c|}{Poměr pleiotropických alel} \\
\hline
\multirowcell{2}{Podíl neg. \\ dom. alel} & \multicolumn{3}{c|}{0.00} & \multicolumn{3}{c|}{0.25} & \multicolumn{3}{c|}{0.50} \\
\cline{2-10}
        & avg & med & s.d. & avg & med & s.d. & avg & med & s.d. \\
\hline
  0.00                        & 0.01073 & 0.01017 & 0.00239 & 0.01069 & 0.01001 & 0.00232 & 0.01034 & 0.01001 & 0.00165 \\
  0.05                        & 0.01065 & 0.01005 & 0.00251 & 0.01055 & 0.01008 & 0.00223 & 0.01041 & 0.00999 & 0.00225 \\
  0.10                        & 0.01061 & 0.01006 & 0.00213 & 0.01053 & 0.01004 & 0.00225 & 0.01037 & 0.00990 & 0.00236 \\
  0.25                        & 0.00910 & 0.00876 & 0.00301 & 0.01060 & 0.01003 & 0.00239 & 0.01059 & 0.00989 & 0.00423 \\
\hline
\end{tabularx}
\caption{Směrnice růstu průměrné fitness na počátku třetího úseku}
\label{table:sm3}
\end{table}

K růstu fitness na počátku druhého a třetího úseku simulace se vztahovaly následující hypotézy:

\begin{enumerate}
    \setcounter{enumi}{4}
    \item{Populace, kde je nenulové zastoupení negativně dominantních a zároveň dostatečné množství alel ovlivňujících
          více složek fenotypu, mají na počátku druhého období větší směrnici vyjadřující růst fitness,
          na konci období menší směrnici růstu fitness.}
    \item{Tento efekt je nastává i po druhé změně optima a je silnější.}
    \item{Tento efekt negativně dominantních alel roste s množstvím alel ovlivňujících více složek fenotypu.}
\end{enumerate}


%%%%%%%%%%%%%%%%%%%%%%%%%%5

\subsection{Konce období}

Základní popisné statistiky pro směrnice růstu průměrné fitness na koncích jednotlivých úseků simulace
jsou zaznamenány v tabulkách \ref{table:sem1} až \ref{table:sem3}. Pro každou je uveden aritmetický průměr, medián a
směrodatná odchylka.

Růsty fitness na koncích úseků simulace jsou zanedbatelné, populace stagnovaly.
Z pohledu na průměry a rozptyly je zřejmé, že u některých populací musela fitness velmi mírně klesat (což při
zpracování dat potvrdila i zde neuvedená minima):

\begin{table}[H]
\scriptsize
\catcode`\-=12
\centering
    \begin{tabularx}{1.0\textwidth}{|c|Y|Y|Y|Y|Y|Y|Y|Y|Y|}

\cline{2-10}
\multicolumn{1}{c|}{}
 & \multicolumn{9}{c|}{Poměr pleiotropických alel} \\
\hline
\multirowcell{2}{Podíl neg. \\ dom. alel} & \multicolumn{3}{c|}{0.00} & \multicolumn{3}{c|}{0.25} & \multicolumn{3}{c|}{0.50} \\
\cline{2-10}
        & avg & med & s.d. & avg & med & s.d. & avg & med & s.d. \\
\hline
  0.00                        & 0.00002 & 0.00001 & 0.00009 & 0.00000 & 0.00001 & 0.00009 & 0.00000 & 0.00000 & 0.00009 \\
  0.05                        & 0.00000 & 0.00001 & 0.00008 & 0.00001 & 0.00001 & 0.00010 & 0.00001 & 0.00000 & 0.00010 \\
  0.10                        & 0.00001 & 0.00001 & 0.00011 & 0.00000 & 0.00000 & 0.00010 & 0.00001 & 0.00001 & 0.00010 \\
  0.25                        & 0.00001 & 0.00001 & 0.00010 & 0.00000 & 0.00000 & 0.00010 & 0.00000 & 0.00000 & 0.00010 \\
\hline
\end{tabularx}
\caption{Směrnice růstu průměrné fitness na konci prvního úseku}
\label{table:sem1}
\end{table}



\begin{table}[H]
\scriptsize
\catcode`\-=12
\centering
    \begin{tabularx}{1.0\textwidth}{|c|Y|Y|Y|Y|Y|Y|Y|Y|Y|}

\cline{2-10}
\multicolumn{1}{c|}{}
 & \multicolumn{9}{c|}{Poměr pleiotropických alel} \\
\hline
\multirowcell{2}{Podíl neg. \\ dom. alel} & \multicolumn{3}{c|}{0.00} & \multicolumn{3}{c|}{0.25} & \multicolumn{3}{c|}{0.50} \\
\cline{2-10}
        & avg & med & s.d. & avg & med & s.d. & avg & med & s.d. \\
\hline
  0.00                        &  0.00002 &  0.00001 & 0.00015 & 0.00002 & 0.00002 & 0.00018 & 0.00002 & 0.00001 & 0.00015 \\
  0.05                        &  0.00002 &  0.00001 & 0.00014 & 0.00001 & 0.00001 & 0.00017 & 0.00001 & 0.00001 & 0.00017 \\
  0.10                        &  0.00000 &  0.00000 & 0.00017 & 0.00002 & 0.00001 & 0.00017 & 0.00001 & 0.00001 & 0.00017 \\
  0.25                        & -1e-5    &  -1e-5   & 0.00018 & 0.00001 & 0.00001 & 0.00019 & 0.00002 & 0.00001 & 0.00018 \\
\hline
\end{tabularx}
\caption{Směrnice růstu průměrné fitness na konci druhého úseku}
\label{table:sem2}
\end{table}

\begin{table}[H]
\scriptsize
\catcode`\-=12
\centering
    \begin{tabularx}{1.0\textwidth}{|c|Y|Y|Y|Y|Y|Y|Y|Y|Y|}

\cline{2-10}
\multicolumn{1}{c|}{}
 & \multicolumn{9}{c|}{Poměr pleiotropických alel} \\
\hline
\multirowcell{2}{Podíl neg. \\ dom. alel} & \multicolumn{3}{c|}{0.00} & \multicolumn{3}{c|}{0.25} & \multicolumn{3}{c|}{0.50} \\
\cline{2-10}
        & avg & med & s.d. & avg & med & s.d. & avg & med & s.d. \\
\hline
  0.00                        & 0.00000 & 0.00000 & 0.00009 & 0.00000 & 0.00000 & 0.00008 &  0.00000 &  0.00000 & 0.00009 \\
  0.05                        & 0.00000 & 0.00000 & 0.00008 & 0.00000 & 0.00000 & 0.00008 &  0.00000 &  0.00000 & 0.00008 \\
  0.10                        & 0.00001 & 0.00001 & 0.00009 & 0.00000 & 0.00000 & 0.00010 &  0.00000 &  0.00000 & 0.00009 \\
  0.25                        & 0.00000 & 0.00000 & 0.00011 & 0.00000 & 0.00000 & 0.00009 &  -1e-5   &   -1e-5  & 0.00010 \\
\hline
\end{tabularx}
\caption{Směrnice růstu průměrné fitness na konci třetího úseku}
\label{table:sem3}
\end{table}

