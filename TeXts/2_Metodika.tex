\chapter{Metodika}


%V této části buďte velmi precizní. Vše musíte popsat tak důkladně, aby kdokoliv mohl vaše pokusy či
%pozorování zopakovat. Nezapomeňte uvést velikost vzorků, věk a pohlaví zvířat, denní a roční dobu,
%podmínky chovu nebo popis lokalit (třeba včetně mapky), specifické přístrojové vybavení
%(pokud je to důležité, např.~pro srovnání výsledků s publikovanými údaji, uveďte i přesný typ
%přístroje) a jiné podrobnosti. Také může být dobré zmínit, jak jste zabránili vlivu pakovaného testování stejných
%individuí a proč se domníváte, že je počet jedinců dostatečný k zodpovězení vašich otázek. Více než
%vhodné je zmínit a zdůvodnit použité statistické metody a počítačové programy. Používáte-li zkratky, uveďte
%jejich seznam.
%Materiál a metodika bývají pro větší srozumitelnost často členěny na menší podkapitoly: materiál,
%experimentální design, analýza dat atd. Systém více podkapitol třeba jen o třech řádcích bývá přehlednější
%než souvislý odstavec na dvě strany. (A to samozřejmě neplatí jen pro metodiku.)
%Čtení DP usnadní, pokud je členění na podkapitoly obdobné v kapitolách Materiál a metodika,
%Výsledky a případně i Diskuse.
%Při vší pečlivosti se však snažte být poměrně struční. Materiál a metodika by neměly tvořit většinu DP.

XXX neco trosku obecnejsiho?

\section{Simulace}

Použil jsem stochastický model postavený na individuích. Individua jsou jedinci jednoho druhu s oddělenými
pohlavími a žijí v nestrukturovaném prostředí. Fenotyp jedinců i optimum prostředí jsou čtyřrozměrnými
vektory. Optimum prostředí $E(t)$ není stálé, ale může se měnit v čase. Euklidovská vzdálenost fenotypu jedince
od aktuálního optima prostředí určuje jeho fitness.

Optimum prostředí je po dobu 2048 kroků simulace konstantní a následně dojde k jeho skokové změně
o $\Delta{}E$. Po této změně opět zůstane optimum neměnné.

Počáteční velikost populace je jedním z parametrů simulace. V~dalších krocích je následně velikost
populace udržována mechanismem turbidostatu. Navíc je každý jedinec limitován maximální délkou života -- po 64
krocích umírá. Jedinci jsou sexuálně dospělí ihned v následujícím kroku simulace a věk neovliňuje jejich fenotyp.

V každém kroku jsou z populace náhodně vybrány páry opačného pohlaví a osmina z nich se může rozmnožit. Kolik
vyvedou potomků, je určeno průměrnou fitness páru. Způsob, jak je jsou určeny počty potomků, jejich geny a
jak následně geny určují jejich fenotyp je popsán později.
Při každém kroku simulace také velmi malou část jedinců postihne mutace jednoho nebo více genů.

\section{Jedinec a jeho geny}

Geny každého jedince jsou uloženy v dvou homologních chromosomech. Každý z těchto chromosomů má 50 locusů.
Na každém z locusů může být jedna ze třech variant genu - možné alely pro oba \textit{i}-té locusy jsou
značeny $G_i{}1$, $G_i{}2$ a $G_i{}3$ (dále obvykle značeny jen jako \textit{G1, G2, G3}, pokud
nemůže dojít k záměně).

\begin{tikzpicture}
\tikzstyle{every path}=[very thick]

\edef\sizetape{0.7cm}
\tikzstyle{tmtape}=[draw,minimum size=\sizetape]

\begin{scope}[start chain=1 going right,node distance=-0.15mm]
    \node [on chain=1,tmtape,draw=none] {$\ldots$};
    \node [on chain=1,tmtape] {G1};
    \node [on chain=1,tmtape] {G1};
    \node [on chain=1,tmtape] {G2};
    \node [on chain=1,tmtape] {G3};
    \node [on chain=1,tmtape] {G3};
    \node [on chain=1,tmtape] {G1};
    \node [on chain=1,tmtape,draw=none] {$\ldots$};
    \node [on chain=1] {\textbf{Chromozom 1}};
\end{scope}


\begin{scope}[shift={(0cm, - \sizetape)},start chain=1 going right,node distance=-0.15mm]
    \node [on chain=1,tmtape,draw=none] {$\ldots$};
    \node [on chain=1,tmtape] {G2};
    \node [on chain=1,tmtape] {G2};
    \node [on chain=1,tmtape] {G2};
    \node [on chain=1,tmtape] {G2};
    \node [on chain=1,tmtape] {G3};
    \node [on chain=1,tmtape] {G1};
    \node [on chain=1,tmtape,draw=none] {$\ldots$};
    \node [on chain=1] {\textbf{Chromozom 2}};
\end{scope}
\end{tikzpicture}

\subsection{Mutace}


\subsection{Křížení}



\section{Jedinec a jeho fenotyp}

Fenotyp jedince jsou jeho čtyři vlastnosti dohromady reprezntované jako čtyřrozměrný vektor.


\section{Řízení velikosti populace}


\section{Frozen Beagle}

Popsaná simulace byla implementována v silně staticky typovaném čistě funkcionálním jazyce \cite{haskell}.

\url {https://github.com/satai/FrozenBeagle}

\section{Statistická analýza}
\section{Reprodukovatelnost}

Bla, bla.

K sazbě byl použit systém \TeX se sadou maker \LaTeX a za použití fontů Latin Modern.
