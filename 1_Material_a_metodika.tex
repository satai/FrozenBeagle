\chapter{Metodika}

\section{Simulace}

V této části buďte velmi precizní. Vše musíte popsat tak důkladně, aby kdokoliv mohl vaše pokusy či pozorování zopakovat. Nezapomeňte uvést velikost vzorků, věk a pohlaví zvířat, denní a roční dobu, podmínky chovu nebo popis lokalit (třeba včetně mapky), specifické přístrojové vybavení (pokud je to důležité, např. pro srovnání výsledků s publikovanými údaji, uveďte i přesný typ přístroje) a jiné podrobnosti. Také může být dobré zmínit, jak jste zabránili vlivu pakovaného testování stejných
individuí a proč se domníváte, že je počet jedinců dostatečný k zodpovězení vašich otázek. Více než vhodné je zmínit a zdůvodnit použité statistické metody a počítačové programy. Používáte-li zkratky, uveďte jejich seznam.
Materiál a metodika bývají pro větší srozumitelnost často členěny na menší podkapitoly: materiál, experimentální design, analýza dat atd. Systém více podkapitol třeba jen o třech řádcích bývá přehlednější než souvislý odstavec na dvě strany. (A to samozřejmě neplatí jen pro metodiku.) Čtení DP usnadní, pokud je členění na podkapitoly obdobné v kapitolách Materiál a metodika, Výsledky a případně i Diskuse.
Při vší pečlivosti se však snažte být poměrně struční. Materiál a metodika by neměly tvořit většinu DP.

\section{Frozen Beagle}
Název první kapitoly
\section{Statistická analýza}
\section{Reprodukovatelnost}
