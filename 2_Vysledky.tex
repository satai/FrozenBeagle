\chapter{Výsledky}

V DP uveďte všechny zjištěné výsledky včetně výsledků pilotních experimentů. Důkladně však zvažte, jaká část vašich výsledků má být prezentována v kapitole Výsledky. Měly by tu být především výsledky, které čímsi přispívají k zodpovězení vašich otázek. Základní data (např. tabulky naměřených rozměrů nebo genotypy jedinců)  patří spíše do Příloh nebo na přiložené CD. 

Mějte na paměti, že obrázky spíše vyjadřují myšlenky, zatímco tabulky zobrazují data. Údaje z tabulek neuvádějte znovu v textu. Pozor také na duplikaci údajů obsažených v tabulkách a grafech. Každá tabulka a graf však musí být v textu zmíněny (zjednodušeně řečeno: tabulka ukáže data, která jsou dále okomentována v textu).

Příklad: V tabulce číslo 1 jsou naměřené délky žížal v cm (10  10,1  10,7  10,9  11).  V textu se pak například objeví: ...z Tabulky 1 vyplývá, že žížaly měřily od 10 do 11cm...

Tabulky i obrázky by měly být očíslovány a podle čísel taky seřazeny (tedy Tabulka 1 se v textu objeví dříve než Tabulka 2).

Věnujte také zvýšenou pozornost popiskům obrázků a tabulek, které musí být \uv{self-explanatoryi}  a zkontrolujte popis os v grafech. Tabulky by měly být co nejjednodušší. Vertikální čáry vnich nejlépe nepoužívejte vůbec a počet čar horizontálních omezte na minimum.

Musí být zřejmé, nejen které statistické testy byly použity, ale také zda jsou pro ně splněny předpoklady (např. normální rozložení, pokud to test vyžaduje). 

Samozřejmě můžete použít barevné grafy či tabulky. Je ale dobré si uvědomit, že se hodnotí obsah, nikoliv barevnost. Klidně tedy vystačíte i jen s černou a bílou barvou (to samozřejmě nemusí platit pro fotografie nebo obrázky). Pokud použijete barevné grafy, měly by být (pokud možno) rozlišitelné i v černobílém provedení.

\section{Název první podkapitoly v druhé kapitole}

\section{Název druhé podkapitoly v druhé kapitole}

